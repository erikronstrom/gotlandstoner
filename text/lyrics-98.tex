\setlength{\columnsep}{0cm}
\begin{multicols}{2}
2.  Förnögd u fräi fran sorg u kval,
    mi stillä andakt röirdä.
    U fran de lagä jårdis dal
    min fräiä, taŋkä hoirdä.
    Ja täŋŋtä uppa täidens fart,
    lains uŋgdoumsåri äilar snat.
    Sitt mål di hastut hinnar
    u som ’n roik fösvinnar.
3.  Ja täŋŋt uppa dän sällä täid,
    vaäi vö måstä strävä.
    Vör måst jå alltut havä sträid,
    så läŋgä vör hä livä,
    mout hat u avund, synd u flärd,
    mout falskä tuŋgars skarpä svärd,
    mout äŋsel, sorg u räŋkar,
    sum falskä vännar skäŋkar.
4.  Mä släikä taŋkar äi mitt braust
    u känsel äuta smärtå
    ja hoirdä vissläs mä saktä raust
    fra käleiksfullä järten.
    Dä komm ifran täu vännas munn,
    sum sat i summans gröinä lunn.
    Däit vändar ja mitt oirä
    att dairäs samtal hoirä.
5.  Va käleik hä ja hoirdä da,
    sum människjärten givä,
    ai rördä järten kann förma
    ti fylläs daim biskrivä.
    Så jäuveli sum käleik jär,
    nä dän sin framgaŋg vinnar här,
    stäur smärtå dän u givar,
    nä dän föhindräd bläivar.
\vfill\columnbreak
6.  Dän ainä vännen sägdä da:
    »Mitt järtäs älskarinnä,
    mä di ja avske nå ska ta,
    fast mäinä tarar rinnä.
    Mäin släkt u vännar naikä mi
    att läŋgar varä vänn mä di.
    Mi gunst u lykkå sväikar,
    um ja ai fran di väikar.
7.  Ja aldri täŋŋt äi värden här,
    att ja di skuddä mistä.
    Ja ständut täŋŋt di haldä ker
    inti min livnäds säistä.
    Män fräihait, just den ädlä lott,
    dän ha ja ai äi värden fått,
    sum läikväl fäuglen aigar,
    sum fräi äi lufti svävar.»
8.  Ha svarädä mä änslut moud:
    »Ska dättä skei u händä,
    att ourä vänskap jäuv u gou,
    så snat ska ta sin ändä?
    Ha räikädoum så lukkä di,
    att däu så yvargivar mi?
    Ha ärå, gunst u lykkä
    mi tat äutör ditt tykkä?
9.  Ska aldri mair äutäi din famn
    ja nåken gaŋg bläi släuten?
    Ska nå din jäuvä käleikshamn
    fö mi u bläi tisläuten?
    Ska ja ai mair dän hugnäd na
    att äuti ainslihaiti fa
    fötroulit mä di talä
    u järtä mitt häugsvalä?
\vfill\columnbreak
10. Ja, sum a turtudäuå säll,
    sum bour blant sköinä kvistar —
    ha sörgar ti sin livnäskväld,
    nä ha sin makä mistar;
    så gar u ja äi dännä värd,
    ja jär äi käleiksboijar snärd.
    Ä sorglit läiv ja njäutar,
    ti däss min vandriŋg släutar.
11. O skaparä, sum danä mi
    att sorgis offer varä,
    så ta da bort mitt uŋgä läiv
    u lätt mi häden farä
    ifran ’n värd mä ourou full!
    Så ma dokk äi dän tystä muld
    mitt sorgäfullä järtä
    bläi fräit ifran all smärtä.
\vfill\columnbreak
12. Da väilar ja äi rou u frid,
    ti däss mi Härren väkkar,
    u äi de stoft, sum goimar mi,
    säin starkä hand äutsträkkar.
    Da ska väll ja u däu, min vänn,
    varandrä atar säi igänn.
    Dä bär mi nå förnöijar,
    så läŋgä ja hä dröijar.»
\end{multicols}
