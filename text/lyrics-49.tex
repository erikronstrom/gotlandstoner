2.  Ägde också mina små talanger,
    fast man egenkär ej vara bör;
    hade också oftast komplimanger
    av min spegel ock min kurtisör.
    Tiden gick så hastigt, som man drömde,
    ungdomsåren det var blott en dag.
    Den mig tillbad, hastigt mig förglömde,
    ock min spegel glömde även jag.
3.  Sedan jag min rygg åt världen vände,
    smög mig oförmärkt uti en vrå,
    där mig inga oförrätter hände.
    Jag drack kaffe, lärde mig att spå.
    Jag får äran mamsell gratulera,
    jag därför så tidigt kommen är,
    ock för ingen vill mig presentera.
    Jag min kaffekopp i handen bär.
4.  Inom kort så får mamsell stå fadder,
    resa bort, får roligt, se sin vän.
    Om han då får höra lite sladder,
    blir han åter lika glad igän.
    Pängar får mamsell, en duktig summa,
    ock blir säkert gifter med en präst.
    Sen blir hon som jag en gammal gumma.
    Nu har jag spått mamsell, som jag kan bäst!
