\begin{adjustwidth}{-1mm}{-1mm}
\setlength{\columnsep}{0.1cm}
\begin{multicols}{2}
2.  Ȁr det nu sant, som du talat har med mig,
    att du håller mig så järteligen kär,
    då får du gå hem till min hulda far ock mor
    ock begära trolovning med mig.»
3.  »Din fader och moder besökte jag i går,
    de svarade mig alldeles nej.
    Kan inte sköna jungfrun taga råd av sig själv
    ock rymma utur landet med mej?»
4.  »Gärna kan jag taga råd av mig själv
    ock rymma utur landet med dig.
    Men kanhända när vi komma
    \tabto{4cm}på ett främmande land,
    att du då tänker på att svika mig.»
5.  Ȁr du lik Tomas, som tvivelaktig var,
    eller någon annan dyliker man?
    Kan du då icke tro, vad jag talat har med dig,
    när jag lovat dig mitt järta ock min hand?»
6.  När som de kommo uppå främmande land
    bland fränder ock vännerna mång',
    strax fick unger svennen se en fager ung mö,
    ock då började han svika sin vän.
\vfill\columnbreak
7.  Skön jungfrun då nedföll på sina båda knän
    ock bad en bön med tårar på kind,
    att unger sven månd komma
    \tabto{4.5cm}uppå hännes faders gård,
    ock vara både halter ock blind.
8.  När sju runda år voro gångna förbi,
    så hörde Härren sköna jungfruns bön,
    ty då kom han gångandes på kryckorna två,
    ock var både halter ock blind.
9.  \textls[-5]{»Stånder opp, stånder opp, sköna jungfrun så båld,}
    ock giv din forna vän lite bröd!
    Ty jag minns nog den dagen, liksom den var i går
    då jag led varken hunger eller nöd.»
10. »Stånda opp, stånda opp, mina söner två,
    ock giv min forna vän lite bröd!
    Jag minnes nog den dagen, liksom den var i går,
    då du led varken hunger eller nöd.»
\end{multicols}
\end{adjustwidth}
