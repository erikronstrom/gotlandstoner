\setlength{\columnsep}{0.5cm}
\begin{multicols*}{2}
2.  Den allra yngsta sjöman,
    som uppå skeppet var,
    han ville med jungfrun trolova sig,
    så unger som han var.
3.  Unger sven tog ring av fingret,
    satte den på jungfruns hand:
    »Tag den, tag den, min lilla vän,
    den är ett kärleksband.»
4.  »Vad ska för mor min jag säga,
    när jag nu kommer hem?» —
    »Säg så, säg så, min lilla vän:
    du hitta’n i grön äng!»
5.  »För mor min till att ljuga,
    det går visst inte an.
    Men bättre är att tala sant:
    den är ett kärleksband.»
6.  Unger sven han sig bortreser,
    men kommer snart igän.
    Då hålles bröllop uppå gård,
    ock bruden — är hans vän.
7.  I bröllopsssaln han gångar
    att se som brud sin tröst.
    Han dansar med bruden varvet kring,
    sa sen med sakta röst:
\vfill\columnbreak
8.  »Varför är du så bleker,
    som förr har varit röd?» —
    »En annan har mig lockat falskt
    ock sagt, att du var död.»
9.  »Varför är du så bleker,
    varför är du så blå?» —
    »En annan har mig lockat falskt,
    sen du for härifrån.»
10. Unger sven gick utur salen
    ock i en kammare blev.
    Där satte han sig att skriva fort,
    ock skrev det långa brev.
11. När brevet var fullskrivet
    ock timmen var förbi,
    då drog han ut sitt gyllende svärd,
    stack det igenom sig.
12. När blodet börjar rinna
    i strida strömmar ned,
    då öppnar han sin kammardörr,
    bad flickan skulle se.
13. »O, skåden flickor alla,
    ock se, vad synd det är
    att ha två tungor i en mun,
    ock ha två vänner kär!»
\end{multicols*}
