\setlength{\columnsep}{0.5cm}
\begin{multicols}{2}
2.  Träi öirä, träi öirä fö väisar ja tar,
    träi mark fö en häktriŋg av förtiä par.
    Nå ska er ai präutä, fö ja jär ai döir.
    Fran böindar ti härrar mäin kouså ja stöir.
3.  Ja lättar mi noigä mä va ja kann fa,
    at iŋgen ska säi mi jär oufönoigd ga.
    Ja takkar di folki, sum järo mi hult,
    välsignä daim, Härrä, ti täus’ndä pund!
4.  äi floiten an spälar u lullar galant,
    mä dej han föčanar si maŋgen gou slant.
    U sinä ti krougen fönoigdar han gar,
    u sinä av prästen han skriftniŋgi far.
\vfill\columnbreak
5.  Peir Stabbä han läiknar ein läilä smul dvärg,
    fast krafti ha failar äi bain u äi märg.
    Pa stigä han kläivar, nä han ska stäig pa
    bräun hästen, dän stäurä, han räidar uppa.
6.  Peir Stabbä nå släutar säin väiså hailt kort:
    »Fran sudrä ti nådrä ja raisar nå bort.
    Ja siŋgar u spälar, jär lusti u glad.
    Faväll, mäinä vännar, ha takk fö va dag!»
\end{multicols}
