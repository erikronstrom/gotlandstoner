2.  »Inte ha ja nånsin tappä bort min,
    äuten städespäiku kanskä ha tappä bort sin.»
    Härr Aksel drägar mussäpussen undar sitt skinn,
    så laupar han sinä ti städespäiku in:
    »Hoirar däu, städespäiku, städespäiku mäin,
    har däu int tappä bort mussäpussen din?
    För mussäpussen fant ja pa garden.»
\vspace{8mm}
\tabto{0cm}
\parbox{12.5cm}{Så fortsättes visan flera värsar med samma frågor ock svar, blott med
den ändring, att de tillfrågade personerna hänvisa till andra namn:
»städespäiku» förmodar, att »kökespäiku» tappat bort sin; den sistnämnda hänvisar
till »kammarpäiku», ock så undan för undan, tils »lagålspäiku» blirr tillfrågad.
Då svarar hon så:}

\vspace{5mm}
    »I gar va ja äutä u struppädä\super{2} får,
    da slant mussäpussen vi mitt lår.»
\vspace{12mm}
\tabto{0.2cm}\super{1}) \tabto{0.7cm}en större lösficka, kjolsäck
\tabto{0.2cm}\super{2}) \tabto{0.7cm}\textbf{struppä} = då man mjölkar ock får mycket litet i stävan; t.\,ex.\@ då
\tabto{0.7cm}man mjölkar en ko, som håller på att sina, eller ett får.
