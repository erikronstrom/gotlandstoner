\setlength{\columnsep}{0.1cm}
\begin{multicols}{2}
2.  Härr Lagerman han talade till jungfrun alltså:
    »Huru länge vill du mö för mig gå?»
3.  »I åtta runda åren vill jag mö för dig gå.
    Kommer du då inte, visst gifter jag mig då.»
4.  När åtta runda åren voro gångna förbi,
    ingen härr Lagerman skön jungfrun fick se.
5.  Bägge hännes bröder då lade upp ett råd:
    »I år ska vi gifta kär systern vår.»
\vfill\columnbreak
6.  Till hänne nu friade den stolte härr Tord,
    han hade mera guld, än härr Lagerman had’ jord.
7.  Han hade mera guld uppå sin grevliga hatt,
    än som härr Lagerman gav kronan till skatt.
8.  Han hade mera guld uppå sina fingrar små,
    än som härr Lagerman i hela sitt hov.
9.  De reda deras bröllop i fullan burdus,
    fruar ock fröknar de städa brudehus.
\end{multicols}
\begin{adjustwidth*}{-4.5mm}{-4.5mm}
\setlength{\columnsep}{0.6cm}
\begin{multicols}{2}
10. De reda deras bröllop i fullaste fläng,
    fruar och fröknar de bädda brudesäng.
11. De drucko deras bröllop i dagarna två:
    icke ville bruden åt brudesängen gå.
12. De drucko deras bröllop i dagarna tre:
    icke ville bruden åt brudesängen se.
13. De drucko deras bröllop i dagarna fäm:
    icke ville bruden åt brudesängen än.
14. De drucko deras bröllop i dagarna tolv,
    då togo de bruden till brudesäng med våld.
15. De satte då hänne på rödan gullstol,
    brudetärnor drogo av både strumpor ock skor.
16. Då tittar just bruden genom fönstret ut,
    hon lovade sin Härre, hon tackade sin Gud.
17. Då fick hon se flaggan både rödan ock vit:
    \textls[-5]{»Jag tror, att härr Lagerman har själv kommit hit.»}
18. Då fick hon se flaggan både rödan ock blå,
    den hon had’ tillvärkat med sina fingrar små.
19. »Ack om jag nu hade en förtroliger vän,
    som kunde gå till stranden ock komma snart igän!»
20. Strax hännes yngsta broder var hänne så snäll,
    \textls[-5]{han sprang så ned till stranden ock kom snart igän:}
\vfill\columnbreak
21. »God dag, god dag, kära svågern min!
    Hur står det nu till med skön systern din?»
22. »Skön systern min har stått brud med härr Tord,
    han hade mera guld, än härr Lagerman har jord.» %\vspace{0.3cm}\tabto{0.2cm}23 ock 24 är ett upprepande av vv. 7 ock 8.
23. Han hade mera guld uppå sin grevliga hatt,
    än som härr Lagerman gav kronan till skatt.
24. Han hade mera guld uppå sina fingrar små,
    än som härr Lagerman i hela sitt hov.
25. Härr Lagerman han bannade de böljorna blå,
    för det de icke tagit honom livet ifrån:
26. »I åtta runda åren satt jag fången på en ö,
    Gud banne de dagar, då jag icke kunde dö!»
27. Härr Lagerman han sadlade sin gångare röd,
    han red så litet fortare, än fogelen flög.
28. Härr Lagerman nu gick sig i brudehuset in
    med snövitan hand ock med bleknaden kind.
29. Härr Lagerman gick fram ock tog bruden i hand,
    så drogo de in i ett främmande land:
30. »Hälsa nu hem till min förra fästeman:
    först sörjer han mig, ock sen tar han sig en ann.
31. Hälsa nu hem till de goda bröllopsmän:
    först dricka de en skål, ock så gå hem igän!» —
    \qquad{}mens vi roa oss en stund.
\end{multicols}
\end{adjustwidth*}
