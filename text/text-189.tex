\vspace{10mm}

\textit{Beskrivning:} Alla lekande stå eller sitta nära varandra. En
av sällskapet (förmannen) pekar liksom räknande på varje i leken
deltagande, i ordning som de stå eller sitta, under det hela sällskapet
sjunger ovanstående ordramsa. \textit{En} person utpekas (räknas)
för varje fjärdedel av takten. Den, som sist blir utpekad, skall
hörsamma täxtens uppmaning ock gå till »duri» (dörren). Den
till dörren »dömda» utväljer bland sällskapet en, som han förmår
bära på ryggen, samt kommer med sin börda fram till förmannen.
Mellan dessa uppstår alltid följande \textit{samtal} (ej sång):

\begin{multicols}{2}
\begin{flushleft}
\textit{Förm.:} Va ha däu pa rygg?\break
\textit{Svar:} Säkk.\break
\textit{Förm.:} Va ha däu äi säkk?\break
\textit{Svar:} Pusä.\break
\textit{Förm.:} Va ha däu äi pusä?\break
\textit{Svar:} Husä\super{1}.\break
\textit{Förm.:} Va ha däu äi husä?\break
\textit{Svar:} Nystä.\break
\textit{Förm.:} Va ha däu äi nystä?\break
\textit{Svar:} Nål.\break
\textit{Förm.:} Va ha däu äi nål?\break
\textit{Svar:} Stål.\break
\textit{Förm.:} Va ha däu äi stål?\break
\textit{Svar:} Tjägä.\break
\textit{Förm.:} Spriŋg kriŋg, lätt vägä!\break
\textit{Frågas:} Maŋgä slag?\break
\textit{Förm.:} ʃau (t.\,ex.).
\end{flushleft}
\end{multicols}

\textls[-15]{Nu måste den ankommande springa kring på samma ställe
sju slag, allt fortfarande med sin börda på ryggen.}

\vspace{2mm}

Sedan börjar leken på nytt.

\vspace{5mm}

\super{1}) strumpa
\vspace{5mm}
