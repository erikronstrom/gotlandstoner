\thispagestyle{empty}
\vspace*{5cm}
\begin{adjustwidth*}{30mm}{30mm}
\setlength{\parindent}{1.5em}

\textbf{Hamborskan} är en dans som på Gotland varit känd
»litet varstans». Den spelades i min barndom någon gång
»på enskild begäran», då ett ock annat par svängde om med
varandra, under det hela det övriga sällskapet såg på. Enligt
äldre personers berättelser har den aldrig dansats allmänt,
utan endast av några få. Många hamborskemelodier förekomma
ej häller, vilket i sin mån torde visa, att »hamborska»
aldrig varit någon allmänt omtyckt dans. Spelmän,
som kunde spela ett 60-tal polskor ock
ett 50-tal valser, kunde vanligen blott
två eller tre »hamborskor». Somliga kunde blott en. »Hamborskan» har ej
häller så hög ålder på Gotland som polskan. Enligt vad jag hört omtalas,
hade den gotländska ungdomen lärt sig dansa den i början av 1800-talet av sjömän, som lågo i
hamnarna här ock var på ön. Varifrån dessa sjömän voro, har jag ej fått upplysning om.
Kanske dansen ursprungligen kommit från Hamburg (»Hamburgska»?). Melodien gick i \sfrac{3}{4}
takt ock spelades i långsamt valstempo med mycket stark markering på första fjärdedelen i varje takt.

Den utfördes sålunda: Varje par höll varandra om livet
ock snurrade ett »slag» (= cirkel) omkring för varje takt,
under det de småningom gjorde en stor rundel runtkring
hela danslokalen liksom i valsen. Första fjärdedelen i takten
markerades med vänstra fotens främre del, andra fjärdedelen
med samma fots häl, under det man samtidigt svängde sig
ett halvt slag åt vänster; tredje fjärdedelen markerades med
hela högra foten, varvid man fullbordade andra hälften av
»slaget» (cirkeln).
\end{adjustwidth*}

