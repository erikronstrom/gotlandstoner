\setlength{\columnsep}{0.2cm}
\begin{multicols}{2}
2.  Ja nå gar ti stoiparen,
    där ja jaus ska kaupä.
    Da ja gäinäst ti min vänn
    hastut komm ti laupä.
    För att da fa fråg’en till,
    när ’n ti uss kummä vill
    u mä uss spasserä — u mä uss spasserä.
3.  Ja jär visst oulykkäli,
    att ja så läŋge blevar.
    Mäin madamm ha frågar mi,
    um ja na yvagivar.
    Jaugä för madamm ja gär:
    »Ja fikk läŋge väntä där,
    att fa va ja skuddä — att fa va ja skuddä.»
\vfill\columnbreak
4.  Oftä hugg u slag vör va —
    därpa vör ai aŋkar,
    när vör bärä minnäs pa
    ourä vänn i taŋkar.
    Täiden jär da aldri laŋg,
    när vi ɷr spaserargaŋg,
    ja gar bäi hans säidå — ja gär bäi hans säidå.
5.  Nestä sunndag da bläir ai
    ourä matmour haimä.
    Kumm, min vänn, däu ska fa staik,
    för ja jär allenä.
    Ja vi lagä kaffi till,
    sum däu gännä havä vill.
    Vör ska uss trakterä — vör ska uss trakterä.
\end{multicols}
