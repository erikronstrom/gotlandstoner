{
\vspace{0.7cm}
\setlength{\parindent}{1.5em}
Nigdansen utföres på Gotland sålunda:\smallskip{}

Spelmannen börjar ensam, spelande första reprisen, springa på golvet.
Vid andra reprisens början \textit{niger} han mot en person, denne
fattar ta i hans rockskört ock springer efter honom i takt efter melodien.
När andra reprisen börjar igen, stannar spelmannen ock den efterföljande
ock niga mot varandra, varefter den efterföljande, som vi kunna kalla
n:o 2, niger till en i laget; denne blir nu n:o 3 av de springande
på golvet. Så fortsättes med att niga mot varandra, ock för var gång,
som nigningen försiggår, ökas antalet med en person, som tillkommer.
När första reprisen spelas, springa alla. När nu »rumpan» blivit tillräckligt
lång, eller ej flera vilja vara med i dansen, upphör man att hålla
varandra i rockskörtet eller i klädningen, men i stället skall man
hålla varandra i händerna, utom närmaste man efter spelmannen, som
måste hålla i »spelmansrocken» med ena handen ock med den andra handen
i 3:dje mans hand.

Ock nu börja spelmannen ock alla efter honom, springande i takt efter
musiken, »krypa under arm» under alla armar, som äro med i dansen,
vilket skall ske i rätt ordning.

När allt detta är genomgånget, tar vanligen spelmannen med hela sin
»rumpa» efter sig en utflykt på gården; allt under fiolens oupphörliga
gnidande på samma melodi. När sällskapet då slutligen kommer in, så
skall spelmannen »rullas in», d.\,v.\,s.\ hela sällskapet nystas
upp kring spelmannen, tils det hela slutligen har form av en spiralfjäder,
varefter spelmannen under hurrarop hivas i tak ock leken slutar.
\vspace{0.3cm}
}
