2.  »Ja, kom ock gör mig sällskap, vi ska åt marken gå!
    På vägen ska vi talas vid om ting ock saker små.
    Då tiden är olämplig att plocka blommor blå,
    må vi i stället talas vid om allt, som hända må.»
3.  »Med dig att göra sällskap det går visst intet an,
    ty folken det framföra allt för din fästeman.
    De kunna mig förtala ock säga, att jag går
    emellan dig ock honom, att han dig ej kan få.»
4.  Ja, det får bli min egen sak, vad kan det göra dig?
    Ja, då får var ock en också vara god för sig.
    Ja du får gå till dina, där du har varit förr,
    ock jag får lyckan söka allt för en annan dörr.
5.  Den blackige på hästen han glömmes ofta bort,
    den oförtänkta gästen han gjord’ vår mening kort,
    han drog så ut sin lie ock högg den blomman av —
    det var en vacker flicka som lades ner i grav.
6.  Om denna flicka levat i några månar än,
    visst hade man fått veta, vem som var den flickans vän.
    Emellan dessa murar vilar den jag älskat nu —
    det var en vacker flicka, ja hon fick ej stå brud.
7.  Den högste uppå stolen, den vi tillbedja må,
    som råder över solen ock alla stjärnor små,
    han vill oss alla hava som barn hem till sig.
    De gamla de få komma, när det blir deras tid.
