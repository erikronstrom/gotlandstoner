\setlength{\columnsep}{0.5cm}
\begin{multicols}{2}
2.  »Inte får du dö uti kamp eller strid,
    ej häller på sotesängen din,
    men akta dig väl för de böljorna blå,
    att de ej få förråda ditt liv.»
3.  Härr Pedre han gick sig till sjöstranden ner
    ock skjuter sitt skepp ifrån land.
    Ock skeppet det var utav valfiskeben,
    ock masterna likaså.
4.  Ock seglena voro utan skarlakan röd,
    ock ankartågen de voro blå.
    Ock knappen han var utan finaste guld,
    som ej någon i världen kunde få.
5.  Sen skeppet gått ut ifrån hamnen med fart,
    så började det snart till att stå.
    Då kastades lott ock gulltärning så rar,
    att få reda på syndaren svår.
\vfill\columnbreak
6.  Ock lotten han föll på härr Peder,
    att han var en syndare svår:
    »Ock eftersom jag är en syndare svår,
    så kasta mig i böljorna blå!
7.  Mång städer haver jag bränt uti eld,
    mång kloster haver jag dränkt.
    Mång ärlig mans flicka haver jag narrat,
    deras ära haver jag kränkt.
8.  Om någon av eder skulle komma till lands,
    få se något av fostermodern min,
    så sägen, att jag tjänar på konungens gård,
    att jag lever, är glad uti sinn.
9.  Om någon av eder skulle komma till lands,
    få se något av fästemön min,
    så sägen: Han ligger uti böljorna blå!
    Ock bedjen, att hon gifter sig!»
\end{multicols}
