2.  Bounden tou kraku, mäns ha va varm,
    u stuppädä hännä nir äi sin barm.
3.  Av fjädri de täkktä vör allä our häus,
    av augu de gärdä vör drikkäkräus.
4.  Av kytä de saltäd vör tunnar träi,
    vör kund nå int mair kyt pa kraku säi.
5.  Av karpen\super{1} de gärdä vör floudana skipp,
    di bästä sum äuti floudana gikk.
6.  Av tarmar de gärdä vör allä our raip,
    av baini de gärdä vör dyŋgägraip.
7.  Av häudä gärdä vör körkäknapp,
    av näbben gärdä vör drikkätapp.
8.  De kraku var nytti ti maŋgä gou tiŋg:
    av halsen gärdä vör päikar en riŋg.
    
\vspace{5mm}
\tabto{0.2cm}\super{1}) skelettet
