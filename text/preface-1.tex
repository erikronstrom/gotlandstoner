\pagestyle{empty}
\tinyskip
\vfill
\begin{flushleft}
\small{
Copyright \copyright{} 2017 Wessmans Musikförlag AB \\
Notskrivning och layout: Erik Ronström \\
Tryck: Wessmans Musikförlag AB, Visby

\vspace{3mm}

Noterna satta med Lilypond 2.19.54 \\
Texten satt med \LaTeX{} (LuaTeX 0.95.0)

\vspace{3mm}

ISMN: ??? \\
ISBN: ??? \\
Beställningsnummer: ???
}
\end{flushleft}

\newgeometry{top=33mm,left=36mm,right=36mm,bottom=33mm,twoside,bindingoffset=8mm,headsep=1.8cm}
\fancyhfoffset[E,O]{0pt}
\addtolength{\skip\footins}{\baselineskip}
{
%\setlength{\parindent}{1.5em}

%%%%%%%%%%%%%%%%%%%%%%%%%%%%%%%%%%%%%%%%%%%%%%%%%%%%%%%%%%%%%%%%%%%%%%%%%%%%%%%% 

Beträffande täxten\footnote{Detta kapitel är utarbetat i samråd med lektor M. Klintberg ock fil.\@ lic.\@ H. Gustavson.} till folkvisorna ock danslekarna får
jag meddela följande.

\medskip

Enligt av prof.\@ Lundell uttalad önskan skulle jag uppteckna
täxten så noga efter uttalet som möjligt. Jag har
därutinnan varit så samvetsgrann, som jag kunnat, men formen
ock kanske även innehållet i en del visor torde därpå blivit
lidande. Ty språkformen blir efter de sjungandes uttal (i en
del visor) Varken rent svensk eller rent gutnisk, utan ett
mellanting mellan båda, en sorts gutniserad svenska. När
en gammal sjuttiårig eller åttiårig gotländing ur folkets
djupa led talar i obunden stil, begagnar han ännu vanligen
de äkta gamla ordformerna, men så fort han skall föredraga
något i bunden stil, det må vara av vad slag som hälst, väljer
han vanligen svenskan eller oftast en mellanform. De täxter,
som ursprungligen diktats på svenska, bliva sålunda till en
del gutniserade, varemot de täxter, vilka ursprungligen diktats
på rent gutniskt mål, äro mindre utsatta för denna \guillemotright{}moderna\guillemotright{}
gotländska ock vanligen bibehålla de ursprungliga formerna.
Nu finns det i värkligheten icke många visor, som diktats
på ren gotländska, oaktat många gotländingar diktat visor;
man har merendels föredragit att dikta dem på svenska, fast
ej ren svenska, ty därtill har folket ej varit mäktigt.
Dessa visor hava under tidernas lopp förändrats från svensk
språkform till former mitt emellan svenska ock gammal gotländska.
Jag vill här nedan styrka mitt påstående genom några typiska
exempel för att visa, hur orden uttalas på de tre omtalade
sätten:

\bigskip

\centering{
\begin{tabular*}{0.75\textwidth}{@{\extracolsep{\fill} } l l l }
  \small{Svenska} & \small{mellanform} & \small{gotländska} \medskip \\
  bjuda v. & bjäudä & bjaudä \\
  ljus s. & jäus & jaus \\
  njuter pres. & njäutar & njautar \\
  trög adj. & tröigar & traugar \\
  röda adj. & röä & raudä \\
  huvud s. & hudä & haudä \\
  öga s. & ögä, öigä & augä \\
  köpt sup. & kyft & kaupt \\
  gömma v. & göimä & gåimä \\
  hör pres. & höirar & håirar \\
\end{tabular*}
}



}
\restoregeometry
\fancyhfoffset[E,O]{0pt}
\pagestyle{main}
