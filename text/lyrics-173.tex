\setlength{\columnsep}{0.3cm}
\begin{multicols}{2}
2.  Lunsen koird ti Vasstein u skuddä hälsä pa,
    da var han houp mä Bramkvist,
    \qquad{}at han skudd jalp han fla.
3.  Bramkvist gäinäst svartä: »Dei fa däu frågä far,
    um han mäin kou vi föirä, sum ja nå stulä har.»
4.  \textls[-12]{När Lunsen koird fran Väŋgä, da va han läitä full,}
    \textls[-5]{u nä han komm pa vägen, da rammläd kärru kull.}
5.  Äuti Lunsens skaklar dä va ett haplit brak:
    bakjauli hadd ’n fammä, fammjauli hadd ’n bak.
6.  Uppa Lunsens ansikt dä va ein fasli knöil,
    dä hadd ’n ain mark tobak u så ett hästäföil\super{1}.
\vfill\columnbreak
7.  När hästens pannå granar u augu siŋkar in,
    da siŋgar Lunsen noigdar:
    \qquad{}»Dän kampen bläir snat min.»
8.  Kraku bjaud en dalar, u korpen bjaud en plat,
    män Lunsen svard tibakä: »Dei gar äi häusä at».
9.  Nä andrä stakklar grämä si, va di ska fa ti mat,
    da sättar Lunsen fram itt himlandä kytfat.
10. Dän fäugälen, sum kluŋkar, ja han fa bäitä smat
    äuta di gamblä hästar, sum Lunsen havar flat.
    Kluŋk, fäugel, siŋg! för Lunsen han mår väl!
\end{multicols}
\vspace{5mm}
\tabto{0.2cm}\super{1}) hingstföl
