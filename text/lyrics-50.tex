2.  De blommor ock de blader de gör mig ofta glader,
    hälst när jag spatserar mig åt rosende lund.
    Kanhända att det någon förtryter,
    om jag den lilla vännen tar i famnen en stund?
3.  Kanhända att det gläder dig, att mina tårar rinna
    ifrån mina ögon ock neder på min kind?
    Men det är väl det som dig gläder,
    att jag ej älskar pänningar, jag blir visst aldrig din.
4.  Ja, det skall du få tro ock med grämelse få finna,
    att jag kan vara ensam ock leva utan dig,
    att jag kan vara ensam ock allena:
    ty då har jag ingen som oroar mig.
5.  En gång när jag blir döder ock lagder ner i graven,
    kanhända lilla vännen då gråter efter mig?
    Men en gång på den yttersta dagen
    då får du din belöning för ditt handlingssätt mot mig.
6.  Farväl, min hulda fader! Farväl, min ömma moder!
    Farväl, I kära syskon ock så min lilla vän!
    Min resa hon länder åt norden,
    Gud vet, om jag nånsin mer får se dig, lilla vän.
