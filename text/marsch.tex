\vspace*{1cm}
\setlength{\columnsep}{0.5cm}
\begin{multicols}{2}

Marschen begagnades ej sällan av spelmännen, då de exempelvis i spetsen
för en glad ungdomsskara tågade från eller till dans-, lek- eller
bollplatsen ute i grönan äng en sommarafton; då man från danslogen
tågade in i huvud- eller manbyggnaden för att intaga förfriskningar;
då man från »nöistäwu» (nystugan), där dansen försiggick, skulle in
i »vardässtäwu» (vardagsstugan) »u drikkä flipp».\footnote{En kokt dryck, bestående av öl, brännvin, sirap, kryddor, russin,
sönderskurna äpplen m.\,m.} När gubbarna (de gifta karlarna) vid s.\,k.\ »våg» hade sina gillen
på aftonen ock de började få bättre humör (»umhör») än vanligt, bestodo
oftast deras dansnöjen uti att parvis arm i arm med spelmännen i spetsen
under tonerna av en marschmelodi vandra stugan runt med tummen i vädret.
Vid ovan nämnda ock dylika glada, \textit{icke} högtidliga tillfällen
spelades ock sjöngos marscher med livlig rytm (i alla-breve-takt),
liknande fältmusikens marscher, varför dessa melodier även stundom
begagnades att »\textesh ottsa» (dansa sjottis) efter.

Allmännast förekommo marscher vid bröllop. De högtidligaste ock vackraste
begagnades att spela för brudparet, då det tillika med övriga gäster
i högtidlig procession tågade från t.\,ex.\ prästgården, där alla
vagnarna hålla, ock in i kyrkan samt efter slutad vigsel återvände
dit. Den marsch eller de marscher, som då spelades, kallades i \textit{egentlig}
mening \textbf{»brudmarschen»} eller »brudmarscher». Men för brudparet
spelades marscher vid flera tillfällen: då det gick från brudhuset
ut till vagnarna eller hästarna ute på storgården och tvärtom; då
det skulle ut på aftonen ock visa sig för icke-bröllopsgäster, då
det skulle till bordet för att äta eller mottaga brudgåvor o.\,s.\,v.
Marscherna, som då spelades, kallades ock på de flästa ställen »brudmarscher».
Men marscher spelades för flera än brudparet. Då prästen, bruttebonden,
bruttöverskan ock andra framstående personer anlände till brudhuset,
skulle spelmännen gå emot dem ute på storgården, där vagnen stannade,
ock »spela in» envar av dem i bröllopshuset. Likaledes skulle spelmännen
passa på ock spela för envar av nämnda personer en marsch, även då
de lämnade brudehuset eller bröllopshuset. Spelmännen gingo alltid
före den, som så skulle »spelas in» eller »ut». Dessa marscher kallades
med ett gemensamnt namn \textit{bröllopsmarscher}. När brud eller
brudgum, »ungmor» eller »ungfar» vid avfärden tog avsked av de sina
för att följa sin unga make eller maka till sitt nya hem, spelades
särskilda \textit{»avskedsmarscher»}, vilkas melodi var mycket enkel,
men rörande ock gripande.

Ett annat slag av bröllopsmarscher voro \textbf{ridmarscherna.} Dessa
användes, under det en bröllopsskara eller bröllopsstass red eller
åkte till ock från kyrkan, eller från brudgummens hem till brudens,
eller tvärtom. Ståtligt var det att se, då en bröllopsskara kom ridande
med spelmännen i spetsen. Hade en häst varit med ett par gånger, kunde
han snart markera takten ganska bra. I annat fall spelte spelmännen
så, att hästarnas trav kom att överensstämma med melodiens fjärdedelar.
Det var icke någon lätt uppgift för en spelman att med fiolskrinet
hängande på ryggen, stråken i ena handen ock fiolen i den andra samtidigt
spela ridmarschen ock styra hästen. Vanligen sköts det överallt, så
att elden yrde omkring både häst ock spelman. En härreman från fastlandet
uttalade en gång vid åsynen därav sin förundran till spelmannen, huru
han kunde sitta på häst under sådana kinkiga omständigheter. Spelmannen
svarade då: »De jär mikä let, för vör jär vuksnä ihoup».

\textbf{Staik-strik}\footnote{Strik = stycke (musikstycke); staik = stek.}
kallas de musikstycken, som spelmännen föredrogo, då steken vid bröllops\-mål\-tid\-erna
bars in. Melodierna gå i \sfrac{3}{4} takt ock spelas i raskt tempo.
Spelmännen, som förut suttit vid bröllopsbordet ock njutit av de föregående
rätterna tillsammans med de övriga gästerna, gå, när steken ska bäras
in, ut i köket. Där stämma de sina fioler ock ställa upp skaffarna,
både de manliga ock kvinnliga. Med spelmännen i spetsen, spelande
»staikstriket», komma nu skaffarna springande in, hållande takt med
musikens fjärdedelar, svängande sina stekfat, budande dem än här,
än där; men så fort någon gör min av att taga mot dem, svänga de dem
åt annat håll, tills slutligen faten hamna framför bruttebonden ock
bruttöverskan. Under hela tiden detta försiggår gnida spelmännen oupphörligt
strängarna.

Under det steken sedan ätes, spela spelmännen sina bästa stycken,
s.\,k.\ \textbf{bordsstycken.} Därvid föreslår bruttebonden en kollekts
upptagande till spelmännen.

\textbf{»Rundarium»} eller bondspråket »runndál» kallas de stycken,
som spelmännen föredrogo, då skålar druckos vid bröllop. Även då »brudgåvorna»
upptogos vid bröllop till brudfolket, spelades en repris av ett sådant
stycke för en var, som efter frambärandet av gåvor fick sitt »kvitto»,
d.\,v.\,s.\ drack ett glas vid det bord, där brudparet satt.

När slutligen alla givit sina brudgåvor, tackar brudparet gästerna
ock dricker ett glas. Då spela spelmännen ett \textit{högtidligt}
stycke; \textit{aldrig} de ystra rundarierna. När sådana skålar drickas,
skjutes av åskådarna utanför fönstren med gevär ock pistoler, stundom
så nära ock hårt att fönsterrutorna springa i bitar.
\end{multicols}
