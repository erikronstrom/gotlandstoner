\newpage
1.  Ja gikk ti min broudar dän fystä dagen jaul:
    fikk ja itt kåkkelörä hyns.
2.  Ja gikk ti min broudar dän andrä dagen jaul:
    fikk ja täu kåkkelörä hyns.
3.  Ja gikk ti min broudar dän tridä dagen jaul:
    fikk ja träijä gra gäss;
    täu kåkkelörä hyns.
4.  Ja gikk ti min broudar dän fjärdä dagen jaul:
    fikk ja fäirä par sväln,
    träijä gra gäss o. s. v.
5.  Ja gikk ti min broudar dän fämtä dagen jaul:
    fikk ja fämm flagnä får,
    fiiirä par sväin o. s. v.
6.  Ja gikk ti min broudar dän ʃättä dagen jaul:
    fikk ja soui mä gräisanä säks,
    fämm flagnä får o. s. v.
7.  Ja gikk ti min broudar dän ʃauänd dagen jaul:
    fikk ja ʃau åsnar sed,
    soui mä gräisanä säks o. s. v.
8.  Ja gikk ti min broudar dän åtänd dagen jaul:
    fikk ja åtä nöibörnä kör,
    ʃau åsnar sed o. s. v.
9.  Ja gikk ti min broudar dän näiänd dagen jaul:
    fikk ja nui par tamföidä oikar,
    åtä nöibörnä kör o. s. v.
10. Ja gikk ti min broudar dän täiänd dagen jaul:
    fikk ja täijä gåŋgare gra,
    mä gullsalanä pa,
    spitsanä häŋdä laŋt därifran;
    näi par tamföidä oikar o. s. v.
11. Ja gikk ti min broudar dän ältä dagen jaul:
    fikk ja älvä kliŋgändä klukkar,
    täijä gåŋgare gra
    mä gullsalanä pa,
    spitsanä häŋdä laŋt därifran o. s. v.
12. Ja gikk ti min broudar dän toltä dagen jaul:
    fikk ja toll körkar,
    toll altar i var körkå,
    toll prästar fö vart altar,
    toll kappar pa var präst,
    toll bältar um var kappå,
    toll puŋgar pa vart bältä,
    toll pänniŋgar i var puŋg,
    älvä kliŋgändä klukkar o. s. v.
    
\vspace{5mm}
\tabto{0.2cm} V.\@ 13 se ovan.
