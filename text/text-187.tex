\vspace{5mm}
Leken utföres vanligast ute i det fria. Två av de lekande
ställa sig bredvid varandra med händerna tillsammans.
Alla de övriga ställa upp sig på ett led med ansiktena vända mot det
par, som hålla varandra i handen. En var håller i framför varande
persons kläder, ock så sätter nu hela kedjan sig i rörelse, springande
i takt med melodiens åttondedelar, under det nedan skrivna täxt
sjunges. Den i teten springande styr kosan till det par, som står
med händerna tillsammans. Nu springer hela truppen mellan
nämnda par, som lyfta upp händerna så högt, att alla i kedjan
kunna komma mellan de två under deras upplyfta armar. När
man hunnit så långt i täxten, att man sjungit: »vad heter han
eller hon?» slår paret plötsligen ner armarna framför den »han»
eller »hon», som står i begrepp att komma emellan. »Han» eller
»hon» får då uppgiva ett namn på en person av motsatt kön,
varefter paret åter lyfter upp armarna för fri passage. återstoden
av täxten sjunges sedan under ock mellan parets uppsträckta armar.
