\thispagestyle{empty}
\vspace*{2cm}
\begin{adjustwidth*}{19.5mm}{19.5mm}
\setlength{\parindent}{0em}

\textit{Valserna}, vilka under 1800-talets första hälft bestodo
av förhärskande åttondels-figurer, utfördes också liksom polskan
med schvung ock stark markering på första åttondelen i
varje takt. Vanligen speltes två åttondelar i varje \guillemotright{}stråktag\guillemotright{}
utom de två sista åttondelarna i takten, som skulle ha var
sitt \guillemotright{}tag\guillemotright{}.

\vspace{5mm}

Exempel VIII.

\example{8}

Exempel IX.

\example{9}

När sedan på 1840-talet valser började diktade efter Viener-metoden
med första fjärdedelen i takten punkterad, så spelades
den punkterade noten i ett \guillemotright{}stråktag\guillemotright{} (nedstråk) ock de
återstående treåttondelarna i \textit{ett} (uppstråk).

\vspace{5mm}

Exempel X.

\example{10}

Valser med halvnot ock fjärdedel i varje takt fick var
sitt \guillemotright{}stråk\guillemotright{}; halvnoten nedstråk, fjärdedelen uppstråk. Exempel
därpå torde ej behöva företes.

\vspace{18mm}

%Ungefär så, som här ovan i ord ock exempel visats,
%spelade de flästa gotländska spelmän sina valser
%för en 40 à 50 år sedan.

\textit{Ur avsnittet \guillemotright{}Spelsätt\guillemotright{} i inledningen av Gotlandstoner, s.\@ LII--LIII}

\end{adjustwidth*}

