2.  Den fjortonde september en friare där kom,
    han friar till min flicka ifrån rosendelund:
    »Ja, du må tilltaga, som bladen grönska sig —
    tro aldrig, att jag sviker eller öfvergiver dig!»
3.  »Du fattige gosse, vad tänker du på,
    som tänker på en jungfru, den du aldrig kan få!
    Nej förr skall jag trampa föttren tunna såsom blad,
    förrän du den vännen skall få äga ock ha.»
4.  Icke är väl jag uppå pänningar rik,
    men Gudi vare lov, att jag är ärlig man lik!
    Ty här i världen haver man sorg ock besvär;
    så kanske det blir bäst, att jag ensammen är.
