\setlength{\columnsep}{0.2cm}
\begin{multicols}{2}
2. Så gick jag mig uti stallet in,
   klappa' hästen på gråvita länden.
   Ock red så mig sju mil om en natt,
   när som andra så lätt månde sova --
   \qquad{}O, sjung med sorgeligt bröst, min vän --
   \qquad{}när som andra så lätt månde sova.
3. Så red jag mig uti rosendelund,
   där som alla små foglarna kvittra.
   Ock allt, vad de kvittrade, ock allt vad de sjöng,
   så var det, att min käresta var döder.
   \qquad{}O, sjung o.\,s.\,v.
4. Så red jag mig lite längre fram,
   då fick jag höra klockorna ringa:
   »Ock hören, I ringare, I ringaremän:
   vem ringen I den ringningen före?»
5. Så red jag mig lite bättre fram,
   där fick jag se grävarna gräva:
   »Ock hören, I grävare, I grävaremän:
   vem gräven I den graven före?»
\vfill\columnbreak
6. Både ringare ock grävaremän
   de svarade med sorgelig tunga:
   »Det göra vi för en liten mamsell,
   som snart skall i jorden och ruttna.»
7. Så red jag mig ännu vidare fram,
   då fick jag se bärarna bära:
   »Ock hören, I bärare, I bäraremän,
   vem bären I på denna båren?»
8. De svara' ock sade med sorgelig röst:
   »Dig skall ju svaret förunnas:
   vi bära härpå en liten mamsell,
   vars död av klockan förkunnas.»
9. Hännes like ej fanns uti hela vårt land,
   ej häller i sju kungariken.
   Hännes hals var så vit, hännes fingrar så små,
   hännes ögon voro blå som en himmel --
   \qquad{}O, sjung med sorgeligt bröst, min vän! --
   \qquad{}hännes ögon voro blå som en himmel.
\end{multicols}
