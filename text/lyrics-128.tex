2.  Män um nå ja haddä min vaitä uppskuren,
    så skuddä ja malä i dagana träi.
    Män nå ha ja varken harvä ällar furä,
    inn dä kummar upp, da jär vattnä föbäi.
    \qquad{}Män uppa ett ar ja int mair förmår etc.
3.  Män värkä jär färdut, u dä ska snat laupä,
    u stainar di liggar uppa ouä gard.
    Drivä jär ai färdut, u nal ska vör kaupä,
    u ʃäglä dä jär badd av äskä u al.
    \qquad{}Män uppa ett ar etc.
4.  Ja väntar min grannä i morgå ska malä,
    ja jär så happlit räddar, att dä intä gar an.
    Män um han bär haddä ett staup brännväin i korgen,
    så’t vör kund fa ta uss en sup äi ou kvann.
    \qquad{}Män da ma arr trou, nä dä ga äi skou,
    \qquad{}att vör da ska ta uss a blundå mä rou.
5.  U vör släppt pa vattnä, u dii gynntä skvalä,
    gikk sinä så ste u slou sedi uppå.
    Da komm där en kal, drakk äut vargum tar.
    Ja trour, att issä kvänni kummar aldri ti ga.
6.  Ja dä va en pokkar pa hals u pa sträupä,
    sum så mikä vatten kundä släukä äi ’n hast.
    Dän kalen han var värdar a läutvälliŋg säupä,
    så’t han kund bläi smallar u de sum en trast.
    \qquad{}Däu falskä täiran, va komst däu ifran,
    \qquad{}sum kundä äi di släukä ouä hailä kvänndamm?
