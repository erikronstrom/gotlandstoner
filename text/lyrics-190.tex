Nu gå alla flickor, hållande varandra i hand, emot gossarna
2.  U va vi Simmon de sällä,
    u va vi räideligen härrä,
    u va vi söddarmännar allä?
    Nu gå åter gossarna mot flickorna ock sjunga:
3.  U fräijä vi Simmon de sälle etc.
    Flickorna gå emot gossarna ock sjunga:
4.  U va bjaudar Simmon de sällä etc.
    Gossarna gå emot flickorna ock besvara deras fråga med t. ex.
5.  En smid bjaudar Simmon de sällä etc.
    Man börjar på gossen längst till höger.
    Tycker ej flickan mitt emot gossen om anbudet, så gå alla
6.  U naj far Simmon de sällä etc.
    För var gång ett »naj» angives, taga alla gossar i ring,
7.  U arg bläir Simmon de sälle etc.
    Nu börja åter gossarna frieriet enligt v. 3, ock flickorna fråga
8.  En kopparslagare bjaudar Simmon de sällä etc.
    Tycker nu flickan om anbudet, gå de mot gossarna sjungande:
9.  U ja far Simmon de sällä etc.
    För var gång ett ja angives, taga alla gossar i ring, ock under
10. U glad bläir Simmon de sällä etc.
    Så fortgår leken, tils alla gossar fått ja av sina flickor mitt
Märk: »Gullsmiden», som här nämnes, är andra gossen från höger.
(Alltid ett par i sänder!)
