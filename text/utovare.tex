\documentclass[a4paper,english]{article}
\usepackage[T1]{fontenc}
\usepackage[utf8]{inputenc}
\usepackage{babel}
\usepackage{xfrac}
\begin{document}
\setlength{\columnsep}{0.5cm}
\section*{Folkmusikens utövare}
%\begin{multicols}{2}

på Gotland under 1800-talet kunna grupperas i nedanstående
tre avdelningar:

Grupp I: \textit{Musikidkande ståndspersoner}, som jämte
utövandet av konstmusik även för sitt höga nöje på ett
idealiskt sätt föredrogo vår folkmusik, ock som därigenom
direkt eller indirekt blevo lärare för spelmännen ock bevarare
av den gotländska folkmusikens traditioner.

Bland sådana må nämnas: Laurin, senior ock junior, i
Burs ock Dalhem; Romin, senior ock junior, i Visby ock
Stockholm; konsul Kinberg, rådman Herlitz ock bankkamrer
J. E. Ihre i Visby; kyrkoherde M. S. Kolmodin i Vall m.\,fl.

Grupp II: Mer eller mindre \textit{notkunniga spelmän}, som
spelade in- ock utländsk musik, men därjämte även gotländsk
folkmusik.
De mäst namnkunniga bland dessa voro: läraren C. N.
Carlsson i Lärbro; läraren L. N. Enderberg i Endre; fanjunkar
Lindbom i Sanda; lantbrukaren Karl Odin, Kaupe i Fröjel;
hemmansägaren Lars Lagergren, Likmide i Hemse; tullvaktmästaren
Jakobsson på Östergarn; musikanten Ljungberg i Visby m.\,fl.

Grupp III: \textit{Vanliga gehörsspelmän.}

Sådana voro t.\,ex. groddakarlarna i Fleringe, Medbom i
Väte, Hedström i Gothem, Laugrenska släkten (farfar, far ock
son —-- den sistnämnde dock även notkunnig) i Lau, Burs ock
Alva, »Florsen» i Burs, Cedergrenarna (bröderna Nils ock
Detlof) i Vänge, Andersson i Fide, Hammarlund i Öja, Pucksson
å Klintehamn, Albin, Ekese i Ardre, samt Ahlberg i
Lojsta. De flästa här uppräknade återfinnas i registret.
Bland vissångare ock vissångerskor må nämnas: hemmansägaren
Pettersson, Gudings i Eke, som tillika var klarinettblåsare,
skepparen N. P. Ahlström, Klintehamn, ock hustru
Elisabet Olofsdotter, Flors i Burs.

Nu levande spelmän (1927) anser jag mig ej behöva omnämna
eller beskriva. Största delen, ja kanske alla ha på
senaste årtiondena gjort sig kända vid spelmanstävlingarna,
som gång efter annan hållits på ön från 1908, då den första
spelmanstävlingen gick av stapel i Visby, till den senast
hållna å Lojsta slott midsommaren 1926.

\end{document}
