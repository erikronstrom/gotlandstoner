2.  Ja lycklig är den ungdom, som slipper att tjäna
    ock får vara hemma hos sin hulda far ock mor.
    Jag är ej bland dem,
    jag får väl ta mig fram,
    blott att jag med äran kan bära mitt namn.
3.  Folken de hava mig så mycket på sin tunga.
    Jag tycker, det är nog att var en sörjer blott för sig.
    Jag sörjer blott för mig,
    det räcker nog väl till.
    Sedan få de leva uti världen som de vill.
4.  Dig vill jag likna vid böljorna på havet,
    som drivas utav vinden till obekanta land.
    Ock när de hunnit fram
    intill den sista strand,
    ack låtom oss få fara med till livets sälla land!
5.  Denna lilla visa den har jag själver diktat
    blott för en ungdom, som fattig är som jag.
    Ja vem får här vara utan förtal?
    Det är så svårt att leva uppå denna sorgedal.
