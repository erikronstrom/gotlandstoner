\thispagestyle{empty}
\vspace*{5cm}
\begin{adjustwidth*}{30mm}{30mm}
\setlength{\parindent}{0em}

\textit{Kadriljmelodierna} föredrogos med markerad rytm på
varannan fjärdedel. De som här upptecknats i \sfrac{4}{4} takt, borde
således ha ett nyanseringstecken (\includegraphics[width=1em]{include/accent.pdf}) på 1:a ock 3:e fjärdedelen
ock de, som skrivits i \sfrac{2}{4} takt, nyanseras på första
fjärdedelen i varje takt. De ofta förekommande säxtondelarna
togos av mindre vana spelmän i två \guillemotright{}stråktag\guillemotright{}, men de mera
försigkomna togo dem i ett enda \guillemotright{}tag\guillemotright{}, vilket gjorde större
\guillemotright{}effekt\guillemotright{}.

\end{adjustwidth*}

