\setlength{\columnsep}{0.2cm}
\begin{multicols}{2}
2.  Ein uŋgä haddä ja,
    han hadd iŋgä nasar.
    Da sägd folki mi ett gutt rad:
    ja skudd svaip um pasar\super{1}.
    Nä ja da sin tou pasar ifran,
    hadd en läik vakkrä nasar sum ja.
3.  Ein uŋgä haddä ja,
    han var läitä galen.
    Nä ja komm saint haim ein kväld,
    da hadd ’n gärt äi svalän\super{2}.
    Svalu ha va da full mä lourt,
    stäuu var intä bättar stourt.
4.  Ein uŋgä haddä ja,
    han var mikä nättar.
    Han fikk a skaid haitar groit,
    da blai han traks mättar,
    mättar för alltut, döidar sum stain,
    han röird varkän väŋgar ellar bain.
5.  Ein uŋgä haddä ja,
    han hadd inŋgät augä.
    Da släppt ja äut’n saint ein kväld,
    at han skudd ler si flaugä.
    Gäinäst så flaug’n upp out ’n stukk,
    da fikk’n stäurä augå nukk.
6.  Ja vait mi ein vakkar saud\super{3},
    dän jär äuti Loistä.
    Dän ska jag dräŋkä uŋgän äi,
    bärä ja kundä troistä.
    Fast ja nukk far skrubb av min mann,
    dei fa ga, va dei kann.
\vfill\columnbreak
7.  Ja vait mi itt vakkat bou,
    dei jär fullt av rukkå\super{4}.
    Dä sat ja u suŋnädä till
    såsum a mali nukkå\super{5}.
    Bäst sum ja sat där u sav,
    skräiädä nåkän krakrakrakra.
8.  Ja tou mi a braidbrädd skål
    u skudd vask mi väitä.
    Da komm gamblä kattu däit
    u skudd ihäl mi bäitä.
    Dåtli uŋgar har nukk ja,
    män allä så säir di skaift därpa.
9.  Ja vait mi itt vakkat bou
    äutä äi boundens äŋgä,
    män ner ja ska flaugä däit,
    da far ja flaugä läŋgä.
    Ja flaug bad ein dag u tva —
    aldri pa bou ja raidå kund fa.
10. Ja vait mi itt vakkat bou
    äutä äi boundens hagä.
    Ner ja ein gaŋg skudd flaugä däit,
    gikk ä par ban u fagä\super{6}.
    Bani di hald ä haplit skräi,
    så’t ja matt flaugä bouä förbäi.
    \qquad{}Ma ja intä höilä?
    \qquad{}Jå, ja ma höilä.
\end{multicols}
    
\vspace{5mm}
\tabto{0.2cm}\super{1}) trasorna
\tabto{0.2cm}\super{2}) förstugan
\tabto{0.2cm}\super{3}) brunn
\tabto{0.2cm}\super{4}) den förruttnade veden i ihåliga träd
\tabto{0.2cm}\super{5}) gammal, utlevd käring
\tabto{0.2cm}\super{6}) räfsade marken fri från löv om våren
