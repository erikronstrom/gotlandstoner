\setlength{\columnsep}{1cm}
\begin{multicols}{2}
2.  Det var i går på kyrkovägen,
    då vår granne var så trägen,
    frågte mig trojärtelig
    efter dina år ock ålder.
    Jag vet inte, vad det vålder,
    utan han visst älskar dig.
3.  Jag ej annat kunde svara,
    än vad, som man kan erfara,
    att du blir snart nitton år.
    Kerstin bliver nu snart säxton,
    fast hon är ej stor till växten;
    Malin tolv i denna vår.
4.  Samma dag, som Jöns blev födder,
    då blev gamla prosten dödder.
    Jag minns, som det var i går.
    Vårfrudag i år, som kommer,
    som skall falla in i sommar,
    blir han två ock tjugo år.
5.  Du skall hålla rent i huset,
    pottor, pannor, stopet, kruset,
    som det vor’ av älfenben!
    Ugnen skall du också limma,
    så han skall i huset glimma,
    som’ han vor’ av marmorsten.
\vfill\columnbreak
6.  Du skall intet njugger vara,
    icke häller maten spara
    för dem, du har i ditt bröd!
    Se på deras barn, som svälta,
    de där askakakor älta,
    steka dem på eld ock glöd!
7.  Du skall ej på mannen pocka,
    när du ser en göklik docka
    klä sig över råd ock stånd!
    Fast hon har två vita nävar,
    kan hon ej en skjorta väva
    eller ett par strumpeband.
8.  Du skall också hemma bliva,
    ej i främmand’ stugor driva,
    såsom grannens hustru far!
    När hon i sista stugan rände,
    alla svinen då uppvände
    allt det, som i stugan var.
9.  Tänk, att året har tolv månar!
    Sedan får du hyttla, låna
    allt vad du bör \textit{hålla} på.
    Om du moders regler aktar,
    husetavlan väl betraktar,
    säkert blir du lycklig då.
\end{multicols}
