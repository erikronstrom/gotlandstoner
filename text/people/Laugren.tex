\textbf{Laugren, Olof (den yngre).} Denna trevliga person var född i Burs församling 1819.
Hans fader, arbetaren Olof Laugren den äldre, var en av mellersta Gotlands förnämsta spelmän i början av 1800-talet. Liksom Laugren d.\,ä. fick höra ock lära mycket i musikväg av dåvarande kyrkoherden doktor J. Ph. Laurin i Burs, så fick sonen, Laugren d.\,y., i än högre grad åtnjuta nämnda förmåner av doktor Laurins söner. De lärde honom både att spela ock sjunga. Hans kusin, »Florsen», var nästan jämnårig med honom, ock båda tävlade om att vara flitiga åhörare ock lärjungar hos Olof Laurin, som en längre tid tjänstgjorde som vice pastor åt sin fader i Burs. Laugren lärde sig noter ock kunde själv på egen hand taga ut ett stycke efter noter, om det ej var alltför krångligt, något som »Florsen» aldrig hann med. I uppfattning av musik torde Laugren gått djupare än sin kusin, vilken dock trots sitt gehörsspel alltid var honom överlägsen beträffande, utförandet av mera svårspelta stycken. Men då Laugren spelade sina av pastor »Ole» komponerade sekundstämmor till »Florsens» melodier, ock pastor Ole själv tog sin basfiol (violoncell), blev stämningen synnerligen livlig, ock mången gammal musikkännare, som hört dessa tre i tiotals år spela tillsammans, har försäkrat, att något bättre i den stilen ej gärna kunde åstadkommas.
Sedan Laugrenvid trättio års ålder erhållit klockare befattningen i Alva, drog han sig småningom från det offentliga spelandet vid danstillställningar, hälst han då ej längre hade tillfalle att få spela tillsammans med sina två förtrogne, pastor Ole ock »Florsen». Vanlig ock förekommande, glad ock munter, var han icke allenast omtyckt som spelman, utan kanske ännu mera för sina ädla personliga egenskaper. I 60 år till sin död var han klockare i Alva, för vilken tjänst han erhöll guldmedalj. Vid enskilda tillfällen spelade han, då han bads därom. Hans spel var alltid korrekt ock säkert, såsom han en gång lärt sig styckena, ock han gjorde ej om melodierna, som så många göra, snart sagt till oigänkännlighet. Efter honom ha upptecknats en mängd nummer i denna samling. Han har komponerat en vals i C-dur (n:r \href{548}{548} i samlingen), som är mycket vacker ock vittnar om musikalisk smak ock uppfattning. Langren dog vid 91 års ålder 1910. 

224, 225, 234, 238, 245, 247, 256, 261, 275, 282, 286, 299, 306, 309, 310, 318, 333, 337, 342, 387, 434, 457, 464, 492, 510, 513, 530, 531, 542, 548, 557, 639, 669, 673 
