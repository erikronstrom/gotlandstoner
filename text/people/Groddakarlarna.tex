\textbf{»Groddakarlarna»}, en bekant spelande brödratrio, ha utgått från gården Grodda i Fleringe. 

Fadern, Ole, som också spelade fiol, levde under senare hälften av 1700-talet. Han hade tre söner, som alla antogo namnet Godman. Av dessa blev Jakob hemma på gården, Anders slog sig ned vid Valle i Rute ock Lars vid Ars hamnplats ock fiskläge --- alla tre således i närheten av varandra på norra Gotland. Alla älskade de ett glatt lag, trivdes gärna ock väl utomhus. De blevo uppmärksammade av alla samhällsklasser för sitt utmärkta spel, både solo- ock samspel. Utom vid bröllop, som vid den tiden (början av 1800-talet) kunde räcka nästan en hel vecka, spelade alla tre vid enskilda bjudningar, som äldre ock yngre tillställde. När julnöjena började, spelade de uteslutande för ungdomen. Man kallade sådana nöjen för »hopläggningskalaser», emedan därvid varor lades ihop (in natura), man »bjärd äi» (bar i). Det var malt till öl, kaffe ock socker, vetemjöl till bakning av kaffebröd o.\,s.\,v\@. Så hyrde man sig en stuga, dansade, lekte ringlekar, sjöng folkvisor ock hade roligt, icke minst om »Groddakarlarna» voro tillstädes. Anders i Rute blåste flöjt, de två andra bröderna spelade fiol. De ha komponerat många vackra polskor ock valser, t.\,ex.\@ n:r \href{241}{241}, \href{488}{488} m.\,fl. i denna samling. Men sanningen att säga: gårdsbruket skötte de ej så bra. De tänkte för mycket på musiken ock gingo ständigt smågnolande eller visslande. Väl utkomna på åker eller äng, hände det ibland, att lusten till fiolen blev större än till arbetet, varför de i största hast skyndade hem för att spela. Ock varför just då? Jo, nya tongångar, nya låtar voro uppfunna, ock de skulle till följd av bristande kännedom om noter »spelas in» på fiolen, så att de sedan sutte kvar i minnet. För att stärka ljudet hade de s.\,k. understrängar av metall inuti fiolen. 

När konung Oskar den förste 1854 besökte Storugns härregård i Lärbro, som ligger nära Fleringe, sände man bud efter den ende då kvarlevande av spelmans- ock brödratrion, Groddagubben Jakob. Han blev nu anmodad att spela vid en för konungen anställd bal. Konungen, som själv var fiolist, tilltalades genast av hans spel, språkade med honom ock överlämnade med egen hand en fämtioriksdalerssedel åt honom. Detta blev det skönaste minnet i den gamles återstående levnad. 

Handlanden ock riddaren Grubb å Fårösund omhuldade Groddakarlarna ock såg dem gärna hos sig. Ståndspersoner skulle aldrig ha besökt bondkalasen så mycket, som de gjorde, om det icke hade varit för Groddakarlarnas välljudande spel.

241, 348, 488 
