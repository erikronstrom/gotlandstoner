\textbf{Pucksson} var en av de mera egendomliga typerna bland gotländska spelmän.
På senare tiden av sin levnad bodde han ensam i en gammal båtsmansstuga på Klintehamn ock sökte upppehålla sig genom skoflickning ock fjällfiske, men fattigvården fick stundom gripa in, för att han ej skulle lida nöd. Han hade dock varit med om ganska mycket i sin tid. Kunde man blott vinna hans förtroende, fick man höra många lustiga historier, i vilka han själv spelat främsta rollen. Hans högvälvda panna syntes vittna om klokhet, ögonen om skämt ock humor. I sin ungdom ock mannaålder var han mycket anlitad som klarinettblåsare ock skötte sitt instrument särdeles väl. Han var då med överallt på södra ock mellersta Gotland vid bröllop ock andra danstillställningar. Vid bröllopen stälde han till många tokiga upptåg, vilka icke alls skulle passa nu för tiden. Han var av allmänheten känd som ett slags »trollkarl», som kunde »förvända synen på folk». Under bröllopshögtidligheterna trakterade han gästerna icke allenast med tonerna från sin klarinett, utan mellan danserna uppdukade han en mängd befängda historier ock gjorde en massa »trollkonster». Han hade även kunnat spela fiol, ock de stycken, som 1892 ock 1893 upptecknades efter honom, spelade han på fiol. Klarinetten var då redan för ansträngande för honom att blåsa. Fiolen kunde han då också blott med svårighet sköta, sedan han brutit högra armen. Fast han haft en ganska rik repertoar, kunde han vid ett besök 1893 knappast något stycke helt, ty hans minne var försvagat. Med denna gubbe gick en av Gotlands originellaste spelmän ur tiden. Samlaren minns honom dock från sin barndomstid, då han blåste klarinett »på ett mästerligt sätt», som själva pastor Ole en gång sade om honom. Efter honom ha upptecknats n:r 23, 179, 257, 350, 404, 444, 450, 497, 543, 544, 585, 661, 716, 725.

23, 179, 257, 350, 404, 444, 450, 497, 543, 544, 585, 661, 716, 725