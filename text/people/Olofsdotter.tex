\textbf{Olofsdotter, Elisabet}, Flors i Ardre socken den 2 juni 1821 (porträtt vid titelbladet).
Till följd av föräldrarnas tidiga död måste hon redan i barndomen lämna hemmet för att fostras hos sin kusin ock förmyndare Olof Fransén i Gammalgarn. I ett bondhus på den tiden kunde man ej komma i tillfälle att inhämta mycken kunskap, men det som fanns tillgängligt, studerades så mycket mera, ock det var bibel, psalm- ock evangeliebok, katekes ock - folkvisor. Dessa visor voro icke tillgängliga i tryck, utan lärdes genom föresjungning av andra. Som Elisabet Olofsdotter var utrustad med ett utmärkt minne på flera områden, kunde hon också en stor del av bibeln utantill, ock ännu vid 75 års ålder torde ingen kunnat kugga hänne på en psalm i psalmboken eller en fråga i lindblomska katekesen. Under fäm vintrar (åren 1892-96) upptecknade jag inemot etthundra visor ock danslekar efter hännes föresjungning, ock beundransvärd var hännes förmåga att komma ihåg. När långa visor på 10, 20, ja 30 värser förekommo, kunde ibland en värs komma »förrän han skall», ock en annan tvärtom, men hon rättade det snart. Hon har aldrig haft någon visa uppskriven eller avskriven - alla ha lärts efter andras föresjungning utantill, ock de sutto kvar i hännes minne. I ungdomen tjänade hon som piga på flera orter, såsom i Östergarn, Viklau ock Vänge, varigenom hon kom i livlig beröring med en myckenhet av folk. Repertoaren blev därigenom mera rikhaltig. 1852 ingick hon äktenskap med \href{Florsen}{»Florsen»} i Burs. Hon dog 1897. 

1, 2, 3, 5, 6, 7, 8, 12, 15, 16, 17, 19, 24, 25, 27, 33, 34, 36, 37, 38, 39, 40, 41, 42, 48, 50, 51, 53, 54, 56, 58, 61, 62, 65, 66, 67, 71, 74, 77, 81, 82, 85, 86, 87, 91, 100, 103, 104, 110, 111, 114, 115, 117, 118, 122, 124, 127, 135, 137, 139, 140, 142, 156, 158, 159, 164, 168, 173, 181, 184, 203, 278, 402, 410, 415, 473, 481, 495 
