\textbf{Fredin, August} (samlaren själv), f.\@ 1853 ock son till \href{Florsen}{»Florsen»}
i Burs ock \href{Olofsdotter}{Elisabet Olofsdotter}, folkskollärare i Linde på Gotland åren 1874--1902,
sedermera folkskollärare ock kantor i Loftahammar, Småland. Han säger:

»Från mitt 10:de till mitt 18:de år var jag min far följaktlig på alla möjliga tillställningar, där han var spelman. Jag lärde väl känna folklivet ock folkmusiken, som då var i sitt flor. Sedan mitt 18:de år har jag ej uppträtt som spelman vid danstillställningar, utan blott spelat i slutna sällskap eller för att illustrera mina föredrag om gotländsk folkmusik. Har själv komponerat n:r \href{320}{320} ock \href{369}{369}, varjämte åtskilliga n:r upptecknats ur eget minne efter glömda sagesmän.» 

320, 369