\textbf{»Florsen»}, vars rätta namn var Nils Mårtensson Fredin, var född den 30 nov.\@ 1823 å gården Flors i Burs socken. På Gotland har varit brukligt att giva en bonde ett namn efter den gård, varifrån han är. Om t.\,ex. en bonde är från en gård, som heter Ammunde, så har bonden kallats Ammunden, en från Sixarve gård har kallats Sixarven, ock då nu spelmannen här i fråga var från Flors gård, kallades han merendels »Florsen». Utav de på 1890-talet levande spelmännen hade han »varit mäst med» ock »hållit längst ut». Han fick i sin späda barndom en liten barnfiol att roa sig med, ock det dröjde ej länge, innan han utan någons jälp fick fram rena toner ur densamma. Hans morbror, Olof Laugren den äldre, ock hans äldre kusin, \href{Laugren}{Olof Laugren den yngre}, kommo honom sedermera till jälp ock meddelade honom genom förespelning åtskilliga stycken ur deras repertoarer. Noter skulle han försöka lära, men »dä gikk intä». Han hade ej tålamod därmed, emedan han lärde sig ett stycke efter andras förespelning mycket fortare. I ungdomen kunde han ibland ganska korrekt återgiva ett stycke, som ej var alltför långt, sedan han blott hört det en gång spelas. I sin födelsesocken, där han ständigt bott på samma gård, fick han i ungdomen höra god ock fin musik av de båda prästerna \href{Laurin}{Laurin}, far ock son. Den senare spelade ofta med vid bondbröllopen tillsammans med »Florsen» ock Laugren d.\,y\@. Av pastor Laurin fick han lära sig bättre »stråktak», vilket kom honom väl till pass vid utförandet av de många ock besvärliga gotlandspolskorna. Han skulle säkert ha blivit en framstående förmåga inom fiolmusikens område, om han fått studera detta instrument efter alla konstens regler. Anlitad ock eftersökt som spelman var han också i all sin tid. Vid fina härrskapsbröllop ock härrskapsbaler både i staden ock på landet (t.\,o.\,m. hos landshövdingen Horn i Visby), vid 3--400 bondbröllop, vid »bidningskallas» ock simpla lekstugor togs hans konst i anspråk, ock alla voro nöjda med honom. Ty han har aldrig, såsom så många andra spelmän gjort, »tagit öl för ärende». Man har aldrig sett honom berusad, men väl glad. Då han kunde få sköta fiolen i ett glatt sällskap, var han kanske den muntraste ock gladaste av hela sällskapet. Han var då »glad, som en spelman» skall vara. Utav tidens förfinade seder ock vanor var han alldeles oberörd, varföre han i umgänget i mångt ock mycket föreföll origineä kanske enligt åtskilligas omdöme --- original, i god bemärkelse. Han hade bibehållit förfädrens tänkesätt, talesätt ock rättframhet. 

Många ock lustiga voro de historier, han hade att berätta från sina hundratals utflykter i de olika socknarna både i norra ock södra häradet på Gotland. De hade kanske varit värda ett särskilt kapitel. 

En höst var Florsen ute ock spelade på bondbröllop fäm veckor å rad. Huru han kunde stå ut med så mycket nattvak, tycks vara obegripligt, men han ägde också förmågan att kunna sova (eller åtminstone blunda) ock spela samtidigt. Pastor O. Laurin, som väl kände alla stycken, »Florsen» spelade, berättade en gång, att »Florsen» värkligen sov ibland, då han spelade, ock på framställd fråga, om han då ej spelade galet, sade han: »Inte på annat vis, än att han ibland spelade en repris 3-4 gånger ock en annan blott en gång.» Dansa ock spela kunde han också. Flickan slog en duk om ryggen på honom, ock så bar det av. Rida ock spela på en gång gjorde han som en hel karl. Det behövdes värkligen skicklighet i detta avseende för en spelman att samtidigt, som han skulle spela fiol, sköta en bångstyrig häst i ett ridande brudfölje, där skott på skott skrämde upp den ystra fålen, på vilken spelmannen satt. 

Genom sina många utflykter å bröllop, baler, lekstugor, kransgillen m. m. i en 60-70 års tid kom han i beröring med de flästa gotländska spelmän ock lärde av dem nya stycken. I sitt goda musikminne förvarade han den största delen av all den folkmusik, som spelats på Gotland under 1800-talet. Han dog i maj 1907, 84 år gammal (porträtt vid titelbladet). 

I Ny Illustr.\@ Tidning för 1885, n:r 52, finnes en biografin med porträtt av »Florsen», där följande berätttas om honom.: 

»Han har enligt egen uppgift utfört musik på 266 bröllop i 30 socknar. Många »oroligheter ock galenskaper» vet han att ur troget minne berätta. Ett ock annat må här anföras. Vid ett bröllop i Burs red han på återvägen från kyrkan omkull vid Allmunde bro, då krossades fiolen ock han fördärvade alldeles ryggen. Detta hindrade emellertid ej att, sedan »mor» fått smörja ryggen ock han fått låna en gammal fiol, musiken var i full gång igän. Vid ett bröllop i Levide hade man gömt stråken, men Florsen hittade på råd: han såg sig om efter något, som kunde duga till stråke, ock fick sikte på en tobakspipa. Han tog ur skaftet ock strök på ena sidan litet »konfonium», ock se, det blev stråke av! Sedermera fann han den riktiga uppstucken över »himmeln». En annan gång hade man strukit talg på stråken --- då fick ett spanskrör tjänstgöra i stället. Av livets allvar ock bekymmer har ban fått pröva mycket, ock det vill under sådana förhållanden ett särskilt lynne till att kunna vara glad bland de glada. Men han var det ändå för det mästa, ock väl må man på gubben kunna tillämpa, vad som en gång sades om Lotta Svärd: Något tålde han skrattas åt, men mera hedras ändå». 

178, 186, 223, 226, 230, 231, 232, 234, 241, 246, 252, 253, 259, 264, 267, 270, 271, 274, 276, 290, 293, 298, 302, 303, 305, 308, 313, 314, 315, 318, 319, 321, 324, 327, 329, 330, 334, 335, 336, 344, 345, 348, 355, 359, 361, 366, 371, 372, 373, 374, 376, 377, 378, 381, 382, 383, 385, 395, 398, 408, 413, 417, 418, 421, 423, 425, 427, 429, 431, 433, 436, 438, 440, 442, 446, 448, 454, 456, 462, 471, 472, 474, 477, 480, 483, 584, 485, 488, 491, 492, 494, 499, 501, 502, 503, 508, 509, 511, 512, 513, 516, 517, 519, 521, 523, 526, 527, 530, 532, 534, 536, 538, 540, 546, 549, 550, 552, 555, 557, 559, 561, 562, 563, 564, 572, 579, 589, 597, 601, 602, 612, 614, 615, 617, 619, 625, 630, 632, 634, 636, 638, 645, 648, 651, 665, 672, 674, 676, 677, 678, 679, 680, 681, 684, 685, 686, 687, 689, 690, 691, 692, 694, 695, 696, 697, 701, 702, 703, 704, 705, 706, 707, 708, 709, 710, 711, 712, 720, 721, 722, 724 
