\textbf{»Hagebyen»}, en bonde vid Hageby i Etelhem, som levde i slutet av 1700-talet. Han var på sin tid de gotländska spelmännens överhuvud. Hans vackra polskor vittna ännu därom. P. A. Säve berättar i sina Gotländska samlingar (handskrifter i Uppsala Universitetsbibliotek) följande om Hagebyen ock hans son: 

»Lars Hagebyar i Etelhem, 'den gamle spelmannen', som var känd kring hela Gotland ock 'lagt upp' så många polskor ock menuetter, var en 'övergiven' spelman. Han levde blott för sitt spel; men gården skötte han föga, ock det fick gå som det kunde. Hageböiar hade 'lagt upp' tvänne polskor, av vilka han kallade den ena för Västurhagen ock den andra för Austurhagen, emedan han fått dem i huvudet, då han var i dessa hagar. Han var en gång vid Tänglings-sågen, gick tvärs från ock hem, tog fiolen ock spelade en ny melodi, som han då fått i huvudet --- innan dess ville han ej tala med någon människa. Han steg ibland upp mitt i natten, gick i skjortan på golvet ock spelade. Han kunde spela 'älvastriket', som de 'sma undar jårdi' spelade. 

Hans son, Jakob Hagebyar, som var en stor, grov karl, var av faderns kynne samt också en ivrig ock stor spelman; dock var, såsom alla sade, gubben »argare» (vassare) att spela. När gubben ock sonen stundom satte sig till att spela i 'nystugan' ock 'gårdsfolken' stodo där ock lyssnade till, så tyckte alla, att det var gruvligt att höra på, så härrligt spelade de. Jakob brukade understrängar på sin fiol för att stärka ljudet, ock han hade ofta långa tider hos sig någon 'ungdom', som han lärde spela. Ty han kunde allt slags spel, det måtte vara polskor, valser, menuetter, skrattpolskan, 'allvalsen', 'älvåstriket', brudmarscher, 'staikstrik', o.\,s.\,v., ja psalmer, dem han spelade under vigseln i kyrkan. När han var i lag med andra spelemänner ock ville det, kunde de gärna få spela med honom. Men ville han ej ock strök han mössan tillbaka, 'kom ingen med honom', ty då for hans stråke i så brinnande fart, ock tonerna från fiolen rullade så underligt, att alla bara med öppna munnar förundrade sig ock fästade ögonen på Hageby-Jaku. Han satt där liksom för sig själv med huvudet till ena sidan, trampade takten ock såg på en fläck. Ock därföre voro både gubben ock sonen så eftersökta spelemän, att aldrig någon lekstuga eller något bröllop hölls på flera mils omkrets, där ej ungdomen ock alla väntade sin högsta frtijd av Hagebyarnas spel. Också var det vanligt, att folk, som skulle ha bröllop, förr 'togo dagar ut' för högtiden hos Hagebyar än hos prästen, ty fick man ej höra hans fiol, då fick det vara med hela gästabudet så länge. Därför var det vanligt, hälst på höstens tid, att Hageby-karlarna, far ock son, knöto sina fioler i en lärftsduk, redo 'på gata' för att spela på det ena bröllopet efter det andra ock ej kommo hem igen förrän efter en fjorton dagar eller tre veckor.»

272, 387 
