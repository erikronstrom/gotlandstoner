\textbf{Cedergren, Nils och Detlof}, bröder. Huru kär fiolen ännu på 1850-talet var för de unga på vissa trakter, kan man göra sig en föreställning om, då i en enda liten socken som Vänge med en befolkning av mellan 300 ock 400 personer icke mindre än 18 ynglingar samtidigt kunde spela fiol förutom en del medelålders män ock gubbar. Vid gillena behövde man en tid icke anlita någon särskild spelman: man blott turade om med att spela. Ibland kunde det hända, att alla eller åtminstone största delen medtogo sina fioler ock spelade samtidigt. Ock då stod glädjen högt i tak. 

De som i denna socken mäst utmärkte sig för sin spelkonst voro de två bröderna Nils ock Detlof Cedergren, födda å Bjerges gård i Vänge, den förre 1826 ock den senare 1835. I sin ungdom voro de mycket ute tillsammans i socknen ock trakten däromkring ock blevo slutligen ganska ansedda spelmän. Deras spel var lätt, behagligt ock rent som guld. Som medelålders män upphörde de dock med att offentligen uppträda, sedan de slagit sig på handel ock andra affärer. Men ibland hände det, att Detlof, som är här avbildad, tog fram sin fiol ock spelade av järtans lust. Båda bröderna hade ett mycket behagligt sätt. Vid dansgillen ock andra tillställningar kunde de icke allenast uppliva de närvarande med fiolen, utan även med sin viga mun, som genast var slagfärdig ock kvick att bemöta framkastade spörsmål. 
