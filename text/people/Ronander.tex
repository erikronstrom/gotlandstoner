\textbf{Ronander, Olof}, eller »Tunnu» (Tunnan), som han även benämndes, var ättling av en utgrenad spelmanssläkt.
Farbrodern, Olof Persson, Smissarve i Rone, varpå sin tid en ganska ansedd spelman, varom hans polska n:r \href{388}{388} i denna samling bär vittne, likaså var hans morbror från gården Skåls på Näs. Född 1824, ägnade Ronander sig först åt skomakaryrket, men blev sedan båtsman i Eksta ock fick namnet Tunna (gotl. »Tunnu», eg. 'tunnan'), med vilket namn han sedan av de flästa omnämndes, fastän han snart tog avsked från båtsmanstjänsten ock flyttade till Ronehamn. Utom av de förut nämnda släktingarna lärde han sina vackraste stycken av två duktiga spelmän från södra Gotland: Andersson i Fide ock Hammarlund i Öja (se Hammarlunds polska n:r \href{298}{298}). Ronander har spelat i södra tredingens alla socknar. Mäst anlitad var han
dock i hembygden å Ronehamn, där han spelade på härrskapsbalerna, ofta i sällskap med \href{Florsen}{»Florsen»} från Burs. Då Ronanders repertoar omfattade samma stycken som »Florsens», \href{Laugren}{Laugrens}, \href{Lagergren}{Lagergrens} m. fl., har ej särskilt upptecknats något efter honom, ehuru samlaren hörde honom spela mångfaldiga gånger. Hans spel var lätt, mjukt ock rent, ock han hade en behaglig stråkföring. Fastän fattig, så att han flera gånger måste bedja om bröd, höll han sig alltid snygg ock hade ett gott ock tilltalande utseende. På senaste åren förlorade han synförmågan, men spelade ändå ock deltog som nestor med liv ock lust i spelmanstävlingen i Visby 1908. Trots sina då fyllda 84 år ock fastän blind var han en av dem som renast ock med största bifall utförde sina svåra säxtondelspolskor. Han dog nära 90 år gammal. 

582 
