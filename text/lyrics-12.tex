2.  »ʃymännar vi ja älskä, ʃymännar vi ja ha,
    ʃymännar vi ja älskä, ti däss ja läggs i grav.»
    Män fadann gäinäst svardä: »Däu skall en intä fa,
    i mårgå ska ja lättä den ʃömann ijälsla.»
3.  Män saint um ’n aftän, när ʃymannen komm,
    han klappar uppa pourten mä säinä figgrar sma.
    Han rouptä: »ʃönä flikkä, däu släppar mi väll in?
    Däu vait jå, att ja jär aldräkerästen din?»
4.  Å jåmfrun spragg ör säggi, rykkt låsä ifran,
    u nä ha han fikk säi, bläi ha järtelit glad.
    Ha tou ’an äi sin famn, u ha kysstä han så,
    at allä hännäs sårgar di måndä bortgå.
5.  Um mårgnen mykkä bittit, nä ʃömann skuddä gå,
    stou fadann bakum duri, var påst pa hånnom då,
    så droug ’en äut säin värjå u rändä äi ’äns bröst.
    Hun roupade u skrikade: »Nå mista ja min tröst.»
6.  ʃön jåmfrun hun bad nå sin fadar at fa låv
    i mårgå klåkkän åttä i skoug’n at få gå.
    Ja dä vu hun sökä mussikantana säks,
    ja dä vi hun dansä u hållä si täkk.
7.  Dä drakk hun si ett förgiftit glass väin,
    ja dätta gärdä hun fö aldräkerästen sin:
    »Ja lägg mi äi hans kistå, u sinä äi hans grav,
    så väilar ja förnögdar uppå läilä vännäns arm.»