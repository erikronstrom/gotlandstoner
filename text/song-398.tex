Efter »Florsen» i Burs. Polskan har fått sitt namn efter de svantoner (flageolett),
som här ock var förekomma i de olika repriserna. -- De två första
sammanbundna säxtondelarna i varje fjärdedel i första reprisen
spelas med »uppstråk». Alla \textit{g}, \textit{a}, \textit{h}, och \textit{c} spelas i sammanbindningarna
i första reprisen på d-strängen med fingersättningen 1, 2, 3, 4.
Märk det ovanliga slutet på första reprisen: slutar på \textit{andra}
tonen i oktav i G-dur-skalan.
