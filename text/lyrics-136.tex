2.  Vö måst nå avske tagä fran vännar u ou haim,
    Gu vait, um vör äi täiden fa talä mair mä daim.
    U maŋgä tarar rinnä fran vännens blaikä kinn,
    nä kräigaren ska vandrä fran läilä vännen sin.
3.  Män livar vör u kummar i glädä haim igänn,
    så sägar ouä vännar: »Vällkumnä häit igänn!
    Säg uss, va ha ’r varä u lais ha er nå mått?
    Va nöd ha er ärfarä, u lais ha raisu gat?»
4.  Da kann vör daim bisvarä allt mä ett lustit moud:
    »Ou raiså ha gat lykklit, vör slapp att gjautä bloud.
    Mä dragnä svärd i handi vör alltut färdu stou,
    um ryssen viddä prövä di svänskä gossas moud.»
5.  Vör gossar hä pa Gottlann vö ynskä ai att slåss,
    män um dä skudde gäldä, så finns dä moud äi uss
    u kraftar äuti armen att svärdä föirä ma
    mout ryssen, ouä ouvänn, han skall uss aldri fa.
