2. Den ena hon sade till den andra så här:
   »Varför är du så bedrövad?
   Haver du bortmistat din fader eller mor,
   eller haver du bortmistat din ära?»
3. »Nej, intet har jag mistat varken far eller mor,
   ock Härren Gud bevare min ära!
   Men jag sörjer mäst uppå den fagre unge sven,
   den vi båda älska ock ära.»
4. »Varföre sörjer du på den fagre unge sven,
   den vi båda älska ock ära?
   Han tager väl den, som rikedomarna har,
   den fattige får söka sin like.»
5. Ung svennen han stod just ej långt därifrån,
   han hörde, vad jungfruarna sade:
   »Gud nåde ock styrke mig, armaste man!
   Jag vet ej, vem jag vidare skall taga.»
6. Ung svennen han sig utur buskarna sprang
   ock tog så den fattiga vid handen:
   »Nu är du min, ock jag skall bliva din.
   O Härre, låt oss leva tillsammans!»
7. Äpplet, som växer på högaste gren,
   det måste dock till jordene falla.
   Ock den, som blir besviken utan sin bäste vän,
   han bliver sen besviken av alla.
