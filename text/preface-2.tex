\pagestyle{empty}
\tinyskip
\vfill
\begin{flushleft}
\small{
Copyright \copyright{} 2017 Wessmans Musikförlag AB \\
Notskrivning och layout: Erik Ronström \\
Tryck: Wessmans Musikförlag AB, Visby

\vspace{3mm}

Noterna satta med Lilypond 2.19.54 \\
Texten satt med \LaTeX{} (LuaTeX 0.95.0)

\vspace{3mm}

ISMN: 979-0-66163-536-9 \\
ISBN: 978-91-8771-060-5 \\
Beställningsnummer: 201749A
}
\end{flushleft}

\newgeometry{top=33mm,left=36mm,right=36mm,bottom=33mm,twoside,bindingoffset=8mm,headsep=1.8cm}
\fancyhfoffset[E,O]{0pt}
\addtolength{\skip\footins}{\baselineskip}
{
\setlength{\parindent}{1.5em}


%%%%%%%%%%%%%%%%%%%%%%%%%%%%%%%%%%%%%%%%%%%%%%%%%%%%%%%%%%%%%%%%%%%%%%%%%%%%%%%% 

\section*{\centering \LARGE Förord.}\vspace{1cm}
% \markboth{\MakeUppercase{Förord}}{\MakeUppercase{Förord}}

Kvar att skriva...

% \begin{flushright}
% \textit{Erik Ronström} \\
% \textit{Mästerby, juni 2017}
% \end{flushright}


\clearpage

%%%%%%%%%%%%%%%%%%%%%%%%%%%%%%%%%%%%%%%%%%%%%%%%%%%%%%%%%%%%%%%%%%%%%%%%%%%%%%%% 

\section*{\centering \LARGE Om denna utgåva.}\vspace{1cm}
% \markboth{\MakeUppercase{Förord}}{\MakeUppercase{Förord}}


% Målet med denna utgåva är att den i första hand ska vara \textit{användbar}
% att spela efter.

Utgångspunkten för denna utgåva är att återge originalutgåvan i en ny läsligare
form, men med oförändrat innehåll. I några avseenden skiljer sig dock utgåvorna åt.

\paragraph{Texter.} Originalutgåvan innehåller en över 50 sidor lång inledning,
som ger en kulturhistorisk bakgrund och inramning till det musikaliska materialet.
Där finns inte minst en \guillemotright{}Förteckning över utövare av gotländsk
folkmusik\guillemotright{}.
Av utrymmesskäl har vi dock valt att i denna utgåva bara ta med sådana texter
som direkt relaterar till låtarna och hur de spelats. Dessa avsnitt
har placerats i anslutning till respektive låttyp.

För den som vill läsa mer av Gotlandstoners texter hänvisas till originalutgåvan,
och till hemsidan \href{http://www.gotlandstoner.se/}{www.gotlandstoner.se}, där textmaterialet återfinns i sin helhet.
På hemsidan finns också det förord som Owe Ronström och
Märta Ramsten skrev till Gotlandstoners faksimilutgåva 2004, och som
sätter Gotlandstoner och August Fredin i ett historiskt sammanhang.

\paragraph{Korrigeringar.} De rättelser som finns längst bak i originalutgåvan har
införts i texten och noterna i denna utgåva. Därutöver har minimala redigeringar
gjorts på ett antal ställen (mindre stav- eller tryckfel, felaktiga sidhänvisningar etc).
\guillemotright{}Musikaliska fel\guillemotright{} har rättats utan särskild notis
om de varit \guillemotright{}triviala\guillemotright{}, t.\,ex.\@ när en takt
varit av fel längd på grund av en saknad balk, och det är uppenbart hur det borde ha varit.

Det finns också många ställen i originalutgåvan där man kan \textit{misstänka}
tryckfel i noterna. I dessa fall har generellt inte gjorts någon ändring, och
om någon redigering gjorts är denna tydligt utmärkt.

\paragraph{Balkning.} Balkningen är i denna utgåva omgjord efter den nutida principen
att balkarna i första hand visar slagindelningen.

\begingroup
\widowpenalty 10000
\clubpenalty10000
\paragraph{Taktarter.} Ett fåtal låtar är i originalutgåvan noterade i \sfrac{3}{8},
de har ändrats till \sfrac{3}{4}.
\endgroup

\paragraph{Scordatura.} En del av låtarna är upptecknade efter omstämd fiol, och några av
dessa är i originalutgåvan skrivna i s.\,k.\@ \textit{scordatura}-notation, där noterna
motsvarar fingersättningen på fiolen snarare än de toner som klingar. I denna
utgåva är alla låtar skrivna klingande.

\paragraph{Reprisering.} På några ställen förekommer i originalutgåvan följande mönster:

\vspace{3mm}
\includegraphics{include/snippets/repriser-fel-crop.pdf}
\vspace{3mm}

Det är svårt att göra en entydig tolkning: vart går man när man spelat genom det
andra \guillemotright{}huset\guillemotright{} (2:an)? Jag har valt att tolka det som ett enkelt tryckfel, och
i stället skrivit:

\vspace{3mm}
\includegraphics{include/snippets/repriser-ratt-crop.pdf}

}
\restoregeometry
\fancyhfoffset[E,O]{0pt}
