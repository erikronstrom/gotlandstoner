2.  Källiŋgi tou sin svärvädä rukk
    u dorvädä sin mann i pannen:
    »Däu jär så latar sum ’n gammel stukk,
    män rännar läikväl till ’n annen.
    Däit ja bir di ga, då ska däu ju löidä,
    för ja jär jå dän, sum matmour ska haitä.»
3.  Gubben han gikk ej hässelgarden sin
    u skar si dä täu käppar väitä.
    Källiŋgi svärd för läiv u för döid:
    »Ja, daim skatt du ʃälv fa släitä.»
    Dän ainä av aik, dän andrä av lind,
    u’ daim dägdä han pa danekvinnen sin.
4.  Bani di uppa uŋnen kraup,
    u kattar joug undar gröitu.
    Män mitt uppa golvä haldes dä tiŋg
    så skinnfliŋgar flaugdä väggar umkriŋg.
5.  Bounden han gikk si äutum säin dur,
    där möitar han sin grannäs kvinnä.
    »I nat har ja laikt mä källiŋgi mäin,
    så’t bäggä mäinä augå rinnä.»
6.  »Har du laikt mä källiggi däin så i nat,
    så’t bäggä däinä augå rinnä?
    Har ha pälsä upp ditt gamblä töiväskinn,
    så gärd ha sum a ärli kvinnä.
    Så ska var källiŋg gärä bäi sin;
    ja ska u ga haim u gärä så bäi min.»
    För aj-ja! U dä sägdä ha, u dän sid förgät ha aldri.
