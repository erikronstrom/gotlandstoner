\setlength{\columnsep}{0.2cm}
\begin{multicols}{2}
2.  Vad gäller, när utur det jordiska gruset
    min kropp skall framkallas med fröjd,
    att taga sin palm i det himmelska huset,
    dit själen förut är upphöjd,
    om icke min otrogne vän
    mig känner ock finner igän
    av fruktan för åsyn av rätte dommaren.»
3.  Så hördes en vålnad från grifterna ropa
    till mina föräldrar ovan jord.
    Hav tack för var dag, som vi levat tillhopa
    ock jag blivit uppfödd vid ert bord,
    för det jag i hela tjugo år
    blev vårdad med omsorg så svår,
    fast eder belöning blev svidande sår!
4.  Uppammad uti edert kärliga sköte
    tillsammans med bröderna två,
    dem himmelen give de bleve mig till möte,
    när vi inför tronen ska stå —
    Gud kröne edert jordiska lopp
    med dygder, gudsfruktan och hopp,
    till dess I från jorden till himlen tagens opp!
\vfill\columnbreak
5.  Jag blev i min ungdom uppfostrad med ära,
    inövad i seder ock vett.
    Vad kristeligt befanns, fick jag frihet att lära
    i andligt ock värdseligt sätt.
    Jag hade ock hugnaden stor
    att spisa vid nådenes bord,
    där tröst ock hugsvalelse för själarna gror.
6.  Min år uti oskuld ock nöje förflöto,
    jag växte ock upprann som en ros;
    förstod ej, att tiden ock värden beslöto
    att driva min glädje sin kos.
    Jag tänkte alls intet uppå,
    vad mig uti tiden skulle gå,
    ock att ett så kort slut all min fröjd skulle få.
    — — — — — — — — — — — —
\end{multicols}
\newpage
\begin{multicols}{2}
7.  En främmande gäst blev inkvarterad i huset,
    han njöt både frihet ock ro.
    Han började smickra med kärlighetskruset,
    han bjöd mig sin ära ock sin tro.
    Mitt nej ökte mera hans håg,
    beständigt han bedjande låg,
    till dess han mig fängslad ock övergiven såg.
8.  Förgäves jag tänkte min ängslan att dölja,
    förgäves jag tryckte min håg,
    ty kvalet av smärtan vill järtat förfölja,
    hälst när jag mig övergiven såg.
    I enslighet begret jag min nöd,
    fick även av föräldrarna stöd,
    men kvalet ock smärtan blev likväl min död,
9.  I aderton hundra ock tjugonde året,
    januari den tjugoförsta dag,
    då slapp jag det dolda, men brännande såret,
    när klockan slog ällova slag.
    Då anden utrann ur min kropp
    ock togs ibland änglarna opp.
    Till himlen, dit hade jag mitt endaste hopp.
10. Föräldrarnas sort kan ej någon beskriva,
    fastän jag nu bortgången var.
    Kanhända med tiden ånyo upplivas
    dess sin av »Olivia»-plantan kvar.
    Så liten, så späd ock så ung,
    att öka deras börda så tung!
    Den mödan belöne den himmelske kung!
11. Kanhända, att hon, som på golvet nu dansar,
    förrän I bliven lagda på bår,
    med friska ock sköna »Olivie»-kransar
    kan pryda edert huvud ock hår.
    Kanhända, fast hon nu är svag,
    kan leva den lyckliga dag,
    ock bliva edert nöje ock goda behag.
12. När tiden är ändad, som skall genomtråkas
    för eder på jordenes dal,
    med säkerhet vet jag, vi då skola råkas
    på nytt uti himmelens sal.
    Där sorg ock bekymmer ej mer
    för er eller mig sig tå ter,
    men härlighetens sol där i evighet ler.
\end{multicols}
