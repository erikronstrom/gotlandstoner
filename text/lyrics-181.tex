\tabto{0.2cm}\super{1}) \tabto{0.7cm}bådade
\newpage
\begin{multicols}{2}
Lars: Ja väiskäs, ha så visst
      ha intä aiŋkum\super{2} roupä
      pa gamblä agätak
      u täit pa bani gloupä\super{3};
      män ännu mikä mair
      när ha på körku jär.
      Hädd sägs av gamblum, at
      dei präst u prästu bjär\super{4}.
Peir: Män däu mast trou, min brour,
      mäns öilä var äi kruppen,
      at ha mä skräi u stain
      blai jagi skaiv fran tuppen.
      Da flaug de skanä bort
      u sättäd av mout böin —
      banä\super{5} mi äi stamn —
      dei va a väisam\super{6} söin.
Lars: Dei stakkels kraku fikk
      da visst för ouskuld läidä,
      för samä krakå var
      pa samä dag äi Fäidä.
      Da sägd mäin kunu så:
      »Nå jär de sagu sann,
      at proustmour ska ti böin
      u ha ein annen mann.»
Peir: Skoit intä kvinnfolks ård,
      di jaugä mair än kraku.
      U intä orkar ja
      um släikä droimar snakä.
      Män läikväl har ja lust
      at vitä kalens namn,
      sum ourä prästämour
      ska ta så saint i famn.
Lars: Dä bour en kal i böin,
      sum sumbli doktar kallä
      u sumbli sikterer —
      vaim huksar namni allä!
      Han haitar Jannä Päil
      u jär ein gidar kal.
      Vitt däu sum ja, min brour,
      så drikkä vör ens skal.
Peir: Hå samä kal var jär
      bäi uss föir fjourten dagä.
      Prästläikar var en nukk
      män haddä iŋgen kragä. —
      Ja, kragä far en snat,
      bouklerdar jär en bra.
      Visst mattä han bläi präst,
      um han ska proustmour ha.
\vfill\columnbreak
Lars: Dei jär väl så, min brour;
      män dei jär u så märklit,
      at doktar kann bläi präst.
      Män dei jär u så värklit,
      at Petrus kragä fikk,
      sum bär en ʃoukal var,
      u däu blaist klukkarä,
      sum ʃägläd atä ar.
Peir: Sant Peir var ann för kal
      än däu u ja, däin krykkä.
      Män läikväl jär de sannt,
      så mikä kann ja tykkä,
      at han kann bläi ein präst,
      män aldri bläi čokk
      sum ourä proustäfar
      u fyllä upp hans rokk.
Lars: Nukk bläir han čokkar snat,
      när proustmour far en skoitä
      u bröä vargum dag
      äi gamblä öilä bloitä.
      Dei ha nukk maŋten gaŋg
      gärt gutt äi ourä krupp,
      fast däu jäst buttenlaus —
      ein tratt föräuten prupp.
Peir: Jäst däu så tetar da,
      ha däinä tunnar runnä?
      Ska ja ha nasäkast
      för de sum mour vi unnä?
      Nai väläst ha sum skäŋt
      mi maŋget stadut räus
      ifran den fystä dag
      ja drakk mä lädu däus.
Lars: Ja launä hännä Gud
      för bäggä ourä dailä!
      Ja har u njautä gutt
      fran dei ja gyntä krailä
      äi stäuu sum ein valp
      u sum ein flagen an,
      u samä viluhait
      ha ännu mäinä ban.
Peir: Dei jär väl så, min brour,
      dei ha så alltut varä.
      Män monnä mour ännå
      dei färdi skuddä farä
      u koukä barnsölsgroit?
      Dei ska väl söinäs da,
      nä ha äi körku jär
      u ska äi bräudstoul sta.
\end{multicols}

\vspace{12mm}
\tabto{0.2cm}\super{2}) \tabto{0.7cm}utan betydelse [ropat]
\tabto{0.2cm}\super{3}) \tabto{0.7cm}kraxat glåpord
\tabto{0.2cm}\super{4}) \tabto{0.7cm}angår
\tabto{0.2cm}\super{5}) \tabto{0.7cm}bannade
\tabto{0.2cm}\super{6}) \tabto{0.7cm}ledsam (»ve-sam»)
\newpage
\begin{multicols}{2}
Lars: Ja frag, um dei jär sannt,
      sum folki um han sladrä,
      at han kann spundä äut
      allt fail u main bäi andrä
      u kännä, vargum lid,
      at han mä läiten knäiv
      kann sprättä folki upp,
      sum ga äi livnäs läiv.
Peir: Gud fräi uss visst fran dei,
      da bläir visst mour äi farä,
      u vargä gråitliŋgä
      fikk rännä sum ein harä.
      Män dei jär läikväl sannt
      at han kann skaffä bout
      av bloumar, träi u stain
      för allähandä sout.
Lars: Nå, dei bjär väl ihoup,
      för proustmour jär u doktar,
      sum alla gamblä sar
      u slämbä skadar tåktar.
      Di kunnä dailä så,
      at ha tar kvinnfolk an,
      u han tar kalar mout,
      bad sorkar, svänn u mann.
\vfill\columnbreak
Peir: De fa di nappäs um,
      män vör fa läikur saknä
      släik gild u gou proustmour,
      sum kledä bani naknä.
      Vör mistar maŋgt staup öl
      u maŋgt ett gästäbäud.
      De bläir ein slimbur dag,
      nä ha ska standä bräud.
Lars: Far intä jalpä, brour,
      för Guss försöin ska radä.
      Vör ynskä lykkå till
      u lykkå för dum badä.
      Så täit ja hoirä far
      framdails a krakås skräi,
      så huksar ja pa mour,
      de ska sannsagå bläi.
\end{multicols}

