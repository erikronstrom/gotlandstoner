\documentclass[a4paper,english]{article}
\usepackage[T1]{fontenc}
\usepackage[utf8]{inputenc}
\usepackage{babel}
\usepackage{xfrac}
\begin{document}
\section{Redaktionens förord.}
Första upphovet till den publikation av gotländska melodier,
som nu föreligger avslutad, går så långt tillbaka i tiden som till
1893. Vid årets akademiska sommarkurser gjorde jag bekantskap
med folkskolläraren Aug.\ Fredin. Sedan jag efter hand fått en
föreställning såväl om hans musikaliska begåvning ock den rikedom
av på fäderneön gängse melodier, han satt inne med, som hans
personliga redbarhet, uppgjorde vi planer rörande uppteckning av
Gotlandsmelodier ock publikation i landsmålstidskriften. Enligt denna
plan skulle upptecknas alla melodier, som kunde med någon rätt
anses för gotländska, d.\,v.\,s.\ som bland den gotländska allmogen vore
kända ock spelades eller veterligen hade spelats, med uteslutande
blott av sådana som tydligen ginge tillbaka till \textit{modern} konstmusik
ock längs olika vägar sipprat ut på ön.

Melodierna skulle nedskrivas så, som upptecknaren hört dem
spelas, utan några som hälst korrigeringar eller broderier. Tyvärr
är det gängse notsystemet ett mycket dåligt instrument, det kan
endast angiva de allra grövsta konturerna av en på fiol spelad
melodi. Det motsvaras i fråga om taltäxt av det vanliga alfabetet,
ock det finns på musikens område ingen som hälst motsvarighet till
den noggrannhet (visserligen även den blott approximativ), med vilken
t.\,ex.\ det svenska landsmålsalfabetets några hundratal tecken ock
teckenkombinationer omfattande resurser kunna återge en taltäxt.
De oändliga möjligheter till lokal eller personlig variation eller
individualisering, som fiolen (liksom rösten) bjuder den exekverande
tonkonstnären --- fiolens ock strängarnas egenskaper oberäknade ---
genom olika stämning, finger- ock stråkföring, i fråga om tidsindelning,
tonhöjd, tonfärg ock annat --- om allt detta säger oss notskriften
ingenting. Även om man kompletterade den vanliga notskriften
med bitecken för återgivande av mindre intervaller än halvtoner ---
vilket naturligtvis läte sig göra --- så vore därmed föga vunnet.

Något vetenskapligt tillfredsställande ock i praktiken någorlunda
lätthanterlig metod att \textit{ordna} en större melodisamling, är
ännu icke given. För att i någon mån underlätta finnandet av den melodi,
som sökes, ha vi tillgripit den enkla utvägen att inom varje större
grupp ordna melodierna efter noternas tidsvalör. Gäller det t.\,ex.\ 
polskor (\sfrac{3}{4} takt), ställas --- med bortseende från eventuell upptakt
--- först de melodier, vilkas första takt utgöres av en halv ock en
fjärdedel, därefter komma melodier med en halv ock två åttondelar,
en halv, en åttondel ock två säxtondelar o.\,s.\,v.; därefter komma
melodier som börja med en fjärdedel o.\,s.\,v.\ --- Visorna äro
ordnade i grupper efter täxtinnehållet.

Tryckningen har tagit en i ock för sig orimligt lång tid, beroende
på knappheten av de ekonomiska tillgångar, över vilka tidskriften
förfogat i förhållande till höga tryckkostnader --- ock beroende på
att massor av andra bidrag tävlat om plats inom ett inskränkt område.
Sådan samlingen nu föreligger, är den emellertid ett ståtligt monument
över såväl gutarnas musikaliska begåvning som Fredins musikaliska
intresse, kärlek till fäderneön ock energi.

Sedan under åren 1909--1926 av \guillemotright{}Gotlandstoner\guillemotright{} utgivits fyra
häften: 1909:4 = s.\ 1--128, 1912:4 = s.\ 129--232, 1923:3 = s.\ 
273--368, 1926:2 = s.\ 369--496, innehållande 494 melodier, väcktes,
på initiativ av landshövding Roos vid 1931 års riksdag i båda
kamrarna motioner om dels ett statsanslag å 5000 kr.\ till bekostande
av tryckningen av återstående partier av samlingen, dels 2000 kr.\ 
som välförtjänt honorar åt samlaren. Motionen bifölls vid gemensam
votering.

Lll.

\end{document}
