2.  Pa väg’n så möit’n a päikå, o. s. v.
    ä päikå, sum va fäin — ja, ja o. s. v.
3.  »Vartän, vartän, mäin päikå?
    Vartän, däu ståltä måi?»
4.  »Ja ska ga haim ti min fadar,
    sum bour äi lund’n gröin.»
5.  »Va skatt däu gär bäi din fadar,
    sum bour äi lund’n gröin?»
6.  »Dä ska ja di bloumar plukkä
    at bindä kransar av.»
7.  »Vaim ska di kransana havä?»
    »Daim ska min fästman ha.»
8.  Da drägd’n av sitt figgar
    en rigg av gull så röi.
9.  »Ta dän, ta dän, mäin päikå,
    ta dän, däu ståltä måi!»
10. »Behald din skägk u däin gåvå, o. s. v.
    ja dän mouttagar ai, ja ja» o. s. v.
