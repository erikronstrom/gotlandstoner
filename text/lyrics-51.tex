\begin{multicols}{2}
2.  Aldrig komma de nöjsamma dagar,
    som i nöje förflutit med dig.
    Snart skall du ock med oro erfara,
    att du varit så otrogen mot mig.
3.  Liten blomma jag hade planterat,
    vilken invid mitt järta skulle bo.
    Det var en gång, men sen aldrig mera,
    hon för mina ögon så finge gro.
4.  Redan haver du mig övergivit,
    redan du mig för alltid bortglömt.
    Men se bilden av dig har jag skrivit
    i mitt järta ock alltid där gömt.
5.  Ingen är det som mig nu här känner,
    ty för alla en främling jag är.
    Mina ungdomsbekanta ock vänner
    de ä borta, de äro ej här.
6.  Kan väl hända den dagen kan komma,
    då din ungdom har rasat förbi,
    då du med saknad får ångra,
    att du varit så falsk emot mig.
\vfill\columnbreak
7.  Gröna linden jag gick att beskåda
    för att lugna mitt järta ock sinn.
    Då en tanke mig påföll så svåra,
    ock förtvivlan bemäktiga mitt sinn:
8.  »Jag går här ibland stormande ilar,
    ock mitt järta är utan all tröst.
    Jag minnes väl, var gång jag fick vila
    så lugn ock förnöjd vid ditt bröst.
9.  Kvällen stundar, ock solen går neder,
    ock mitt nöje skall slutas i dag.
    Aldrig kommer för mig ibland eder
    någon glad eller nöjsammer dag.
10. Så farväl jag nu önskar att taga
    utav dig, du min älskade mö!
    Utav fruktan att ej sorgen fördraga
    är det säkert, jag snart måste dö.»
\end{multicols}